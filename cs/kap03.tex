\chapter{Návrh}

(pozn.: předpokládám, že tahle kapitola všechno rozebírá na myšlenkové rovině a k jejímu pochopení nejsou potřeba znalosti z kapitoly 2 - do té zabředneme až v příští kapitole při diskuzi implementace)

\section{Stanovení cíle}

\subsection{Jaký chceme výsledek}
\begin{itemize}
    \item \textbf{Zamyšlení}:
        \begin{itemize}
            \item neodsžitelným ideálem, ke kterému se snažíme rámcově směřovat, je zachycení zážitku, co šerm poskytuje v reálném světě
            \item kromě volného mávání mečem, ostatní aspekty co k boji v reálném světě patří, hry zvládají zachytit docela v pohodě -> ovládání meče poskytuje největší prostor k prozkoumání
            \item chceme obecný systém, co nebude potřebovat specifické úpravy pokaždé, když vytvoříme nový typ nepřítele, a co půjde použít k interakci s herním prostředím
            \item nechceme pro meč ručně vytvářet žádné animace (nemáme na to prostředky + to není programátorsky zajímavá činnost)
            \item meč by se měl tvářit v rámci možností rozumně fyzikálně věrně 
        \end{itemize}
    
    \item \textbf{z pohledu backendu by meč měl zvládat plně volný pohyb na všech 6 stupních volnosti}
        \begin{itemize}
            \item zkusíme znovu to, co Die by the Sword, avšak tentokrát vytvoříme otevřenou platformu, na které bude moci jakýkoliv programátor z komunity experimentovat s metodami ovládání (nedoladěné ovládání (a nemožnost ho spravit komunitou) byl hlavní kámen úrazu DbtS)
            \item pokud se nám osobně nepodaří najít opravdu dobré a zábavné ovládání, někomu jinému z komunity se to povést může. i kdyby ne, na hru vždycky půjde napasovat VR ovladač ;)
        \end{itemize}
    \item \textbf{frontend - ovládání pro klávesnici a myš}
        \begin{itemize}
            \item jediná možnost, která zaručí přístupnost pro všechny
            \item mnoho prostoru k experimentálnímu průzkumu
        \end{itemize}
\end{itemize}

\subsection{Jaké nástroje použijeme}

\begin{itemize}
    \item chceme, aby výsledek naší práce byl co nejvíc přístupný pro náhodné lidi, co by ho chtěli využít k navazujícím experimentům či vlastnímu obohacení
    \item \textbf{použijeme herní engine Unity}
        \begin{itemize}
            \item je mature
            \item nejrozšířenější (viz \ref{unityEngineIntroSection}) -> největší šance že se najde jiný vývojář, co pro naší práci najde využití
        \end{itemize}
    \item na všech místech práce chceme používat co nejvíc rozšířené a standardní systémy a knihovny (-> kompatibilita s prací ostatních lidí)
    \item myslet na dobré OOP practices a štábní kulturu
\end{itemize}



\section{Návrh systému}

\subsection{Základní komponenty}

\begin{itemize}
    \item \textbf{Hráč}
        \begin{itemize}
            \item skript co ovládá pohyb hráče a natočení kamery (SwordsmanMovement) - nezajímavá klasika
            \item orientační body podle kterých zpracovávají vstup jednotlivé stavy SwordMovementu (je potřeba aby se pohybovaly relativně vůči tělu hráče)
            \item jednoduchý capsule collider - pro kolize s environmentem apod.
            \item hráčský model (samostatný prefab) - bone rig, nese animátor, na každé kosti collidery co odpovídají části těla - pro kolize s nepřátelským mečem 
        \end{itemize} 
    \item \textbf{Meč} 
        \begin{itemize}
            \item skript co ovládá pohyb meče (SwordMovement)
                \begin{itemize}
                    \item stavový automat
                    \item z editoru se konfiguruje seznam párů klávesa-stav + jeden defaultní stav
                    \item každý stav nese nějaká data konfigurovaná z editoru, další data může získat od instance ISwordMovement, která do něj je injectnuta
                    \item ISwordMovement - poskytuje deskriptor meče, zdroj vstupu, transform držitele meče a metodu MoveSword()
                    \item aktuální stav je ten, jehož klávesa je právě stisknutá - na tom jediném je volán OnUpdate()
                    \item v OnUpdate() stav vypočte žádanou pozici meče a s tou následně na SwordMovementu zavolá metodu MoveSword()
                    \item MoveSword(struct poziceKamSeMečMáPohnout) - pohyb provádí SwordMovement v průběhu času jak sám uzná za vhodné (možné konfigurovat skrz jeho parametry v editoru)
                \end{itemize}
            \item vlastní model meče
                \begin{itemize}
                    \item vyměnitelná podkomponenta, samostatnej prefab
                    \item obsahuje mesh meče, collidery
                    \item poskytuje \textbf{deskriptor-komponentu} ve které jsou uložené transformy odpovídající význačným bodům (ústí čepele, špička, blokovací bod apod.)
                \end{itemize}
        \end{itemize} 
    \item \textbf{Kořen hierarchie} 
        \begin{itemize}
            \item Nese odpovědnost za to, že meč a šermíř budou správně pospojované (jeden s druhým a s objekty co jdou z vnější scény)
            \item Definují se tu odkazy na všechny objekty okolní scény, které meč nebo šermíř potřebují znát (kamera, healthbar, die-screen)
            \item Definuje se tu zdroj vstupu (klávesnice nebo ovládací skript)
            \item Definuje se tu tabulka dalších obecných parametrů, na které se komponenty uvnitř hráče a meče mohou symbolicky odkazovat
            \item Všechny rozdíly, které odlišují \acs{PC} od \acs{NPC}, by se měly dít v tomto jednom objektu
        \end{itemize} 
    \item \textbf{Důsledky tohoto přístupu} 
        \begin{itemize}
            \item Logika ovládání meče a ovládání hráčské postavy jsou dobře oddělené
            \item Logika meče dobře rozšiřitelná - jednoduché přidat nový mod ovládání
            \item Není problém mít ten samý mod několikrát na různých klávesách ale s jinými parametry
            \item Mezi SwordMovement a jeho mody není problém vecpat dekorátory (např. pro nahrávání rozkazů k pohybu meče)
            \item Abstrakce uživatelského vstupu... není problém ho mockovat, simulovat apod.
            \item Nevýhoda - Data hráče a meče nejsou dobře oddělená
                \begin{itemize}
                    \item Detailní model hráče potřebuje mít k dispozici deskriptor meče, aby věděl, jak má napolohovat procedurální animaci rukou
                    \item Orientační body, které používají některé mody meče, potřebují mít konstantní polohu relativně vůči tělu hráče -> musí být potomky jeho transformu, ne meče
                \end{itemize}
        \end{itemize} 
\end{itemize}

\subsection{Myšlenky o mapování uživatelského vstupu}
\begin{itemize}
    \item k dispozici máme myš a klávesnici
    \item musí být pro hráče jednoduše uchopitelné (- ideálně tedy nějaký pěkně spojitý matematický formalizmus, který jsme pro hráče schopní pěkně vizualizovat)
    \item diskuze variant co používalo Die by the Sword:
        \begin{itemize}
            \item \textbf{Klávesnice} - zastaralý koncept, moderní hráč počítá, že k něčemu využije myš; stejně to tehdy byl jenom slabý odvar z toho, co uměla myš
            \item \textbf{Myš} - ze všech možností zvládá největší podmnožinu pohybů; vyžaduje cvik, ale umožňuje kreativitu a flexibilní reakce v nezvyklé situaci
            \item \textbf{Nahrávky} - na první pohled pěkná myšlenka, v praxi však neumožňují flexibilně reagovat na specifickou situaci (hráč je schopen si vést okamžitý myšlenkový přehled o příliš malém množství nahrávek)
        \end{itemize}
    \item výsledek diskuze:
        \begin{itemize}
            \item pro hráče chceme ovládání myší
            \begin{itemize}
                \item jinou možnost než něco s myší, nebo něco s klávesnicí stejně nemáme
                \item myš jediná má potenciál umožnit jemnou a flexibilní kontrolu (klávesnice má konečný počet kláves - konečný počet stavů)
            \end{itemize}
            \item interně implementujeme i přehrávání nahrávek - bude se hodit pro AI nepřítele (dává nám přímou, dobře uchopitelnou kontrolu nad jeho chováním a tedy obtížností hry) (jednoduchý systém, vhodný pro to v budoucnu sloužit jako základ, proti kterému budeme testovat sofistikovanější AI)
            \item nějaké kreativnější nápady než s čím už přišlo DbtS necháme na později. teď chceme mít něco, co už víme, že je základně použitelné, abysme vůči tomu později nové výstřelky mohli porovnávat
        \end{itemize}
    \item jakým formalizmem mapovat polohu myši na polohu meče:
        \begin{itemize}
            \item nevíme jak přesně to na pozadí dělá DbtS
            \item jednoduchý, dobře vizualizovatelný formalismus.. střílet paprsek z kamery z místa kurzoru a protnout ho s nějakým útvarem 
            \item po dlouhé úvaze jsme se rozhodli, že pěkný útvar k protínání je kulová plocha
                \begin{itemize}
                    \item začneme s ní
                    \item buď se ukáže jako ideální
                    \item nebo se od ní odpíchneme abychom se dobrali k něčemu sofistikovanějšímu
                \end{itemize}
            \item DbtS umí pěkně mávat mečem ze strany na stranu. Co ale jiné typy pohybu, co člověk s mečem běžně dělá - např. blokování
                \begin{itemize}
                    \item mávání s mečem naimplementujeme plnohodnotně jako v DbtS 
                    \item bude však možné přepínat na další mody ovládání
                    \item vždy bude aktivní jenom jeden mod - přepínání stiskem příhodné klávesy (např. shift)
                    \item všechny mody berou na vstupu jeden bod - průnik kurzoru s kulovou plochou - a lyší se v algoritmu, pomocí kterého z tohoto bodu určí cílovou polohu meče
                \end{itemize}
        \end{itemize}
\end{itemize}

\section{Shrnutí}
