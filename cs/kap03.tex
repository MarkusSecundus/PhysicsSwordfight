\chapter{Návrh}

\subsection{Stanovení cíle}

\textbf{neodsžitelným ideálem, ke kterému se snažíme směřovat, je dokonalé zachycení zážitku, co šerm poskytuje v reálném světě}

\textbf{meč by se měl chovat fyzikálně věrně}

\textbf{chceme obecný systém, co nebude potřebovat specifické úpravy pokaždé, když vytvoříme nový typ nepřítele, a co půjde použít k interakci s herním prostředím}

\textbf{z pohledu backendu by meč měl zvládat plně volný pohyb na všech 6 stupních volnosti}
\begin{itemize}
    \item zkusíme znovu to, co Die by the Sword, avšak tentokrát vytvoříme otevřenou platformu, na které bude moci jakýkoliv programátor z komunity experimentovat s metodami ovládání (nedoladěné ovládání (a nemožnost ho spravit komunitou) byl hlavní kámen úrazu DbtS)
    \item pokud se nám osobně nepodaří najít opravdu dobré a zábavné ovládání, někomu jinému z komunity se to povést může. i kdyby ne, na hru vždycky půjde napasovat VR ovladač ;)
\end{itemize}

\textbf{frontend - ovládání pro klávesnici a myš}
\begin{itemize}
    \item stanovíme nějakou podmnožinu, v rámci které bude frontend backendu rozkazovat jak má mečem pohnout
    \item  nějaký pěkný formalismus jak mapovat kurzor na polohu meče
\end{itemize}


\subsection{Co k tomu použijeme}

\textbf{herní engine Unity}
\begin{itemize}
    \item je mature
    \item nejrozšířenější -> největší šance že se najde jinej vývojář, co pro naší práci najde využití
\end{itemize}

\textbf{chceme na všech místech práce používat co nejvíc rozšířené a standardní systémy a knihovny} -> kompatibilita s prací ostatních lidí

\textbf{myslet na modularitu}

\subsection{Myšlenky o mapování vstupu}
\begin{itemize}
    \item k dispozici máme myš a klávesnici
    \item musí být pro hráče jednoduše uchopitelné (- ideálně tedy nějaký pěkně spojitý matematický formalizmus, co se dá dobře vizualizovat)
    \item diskuze variant co používalo Die by the Sword:
        \begin{itemize}
            \item \textbf{Klávesnice} - zastaralý koncept, moderní hráč počítá že k něčemu využije myš; stejně to tehdy byl jenom slabý odvar z toho co uměla myš
            \item \textbf{Myš} - ze všech možností zvládá největší podmnožinu pohybů; vyžaduje cvik, ale umožňuje kreativitu a flexibilní reakce v nezvyklé situaci
            \item \textbf{Nahrávky} - na první pohled pěkná myšlenka, v praxi však neumožňují flexibilně reagovat na specifickou situaci (hráč je schopen si vést okamžitý myšlenkový přehled o příliš malém množství nahrávek)
        \end{itemize}
    \item výsledek diskuze:
        \begin{itemize}
            \item pro hráče chceme ovládání myší
            \begin{itemize}
                \item jinou možnost než něco s myší, nebo něco s klávesnicí stejně nemáme
                \item myš jediná má potenciál umožnit jemnou a flexibilní kontrolu (klávesnice má konečný počet kláves - konečný počet stavů)
            \end{itemize}
            \item interně implementujeme i přehrávání nahrávek - bude se hodit pro AI nepřítele (dává nám přímou, dobře uchopitelnou kontrolu nad jeho chováním a tedy obtížností hry) (jednoduchý systém, vhodný pro to v budoucnu sloužit jako základ, proti kterému budeme testovat sofistikovanější AI)
            \item nějaké kreativnější nápady než s čím už přišlo DbtS necháme na později. teď chceme mít něco, co už víme, že je základně použitelné, abysme vůči tomu později nové výstřelky mohli porovnávat
        \end{itemize}
    \item jakým formalizmem mapovat polohu myši na polohu meče:
        \begin{itemize}
            \item nevíme jak přesně to na pozadí dělá DbtS
            \item  jednoduchý, dobře vizualizovatelný formalismus.. střílet paprsek z kamery z místa kurzoru a protnout ho s nějakým útvarem 
            \item  po dlouhé úvaze jsme se rozhodli, že pěkný útvar k protínání je kulová plocha
            \item  ...
        \end{itemize}
\end{itemize}


\subsection{Jak bude vypadat výsledek}


\section{Shrnutí}
