\chapter{Návrh}

\subsection{Stanovení cíle}

(neodsžitelným ideálem, ke kterému se snažíme směřovat, je dokonalé zachycení zážitku, co šerm poskytuje v reálném světě)

(meč by se měl chovat fyzikálně věrně)

(chceme obecný systém, co nebude potřebovat specifické úpravy pokaždé, když vytvoříme nový typ nepřítele, a co půjde použít k interakci s herním prostředím)

(z pohledu backendu by meč měl zvládat plně volný pohyb na všech 6 stupních volnosti)
- zkusíme znovu to, co Die by the Sword, avšak tentokrát vytvoříme otevřenou platformu, na které bude moci jakýkoliv programátor z komunity experimentovat s metodami ovládání (nedoladěné ovládání (a nemožnost ho spravit komunitou) byl hlavní kámen úrazu DbtS)
    - pokud se nám osobně nepodaří najít opravdu dobré a zábavné ovládání, někomu jinému z komunity se to povést může. i kdyby ne, na hru vždycky půjde napasovat VR ovladač ;)

(frontend - ovládání pro klávesnici a myš)
- stanovíme nějakou podmnožinu, v rámci které bude frontend backendu rozkazovat jak má mečem pohnout
- nějaký pěkný formalismus jak mapovat kurzor na polohu meče


\subsection{Co k tomu použijeme}

- herní engine Unity (mature; nejrozšířenější -> největší šance že se najde jinej vývojář, co pro naší práci najde využití)
- chceme na všech místech práce používat co nejvíc rozšířené a standardní systémy a knihovny (-> kompatibilita s prací ostatních lidí)
- myslet na modularitu

\subsection{Myšlenky o mapování vstupu}
- k dispozici máme myš a klávesnici
- musí být pro hráče jednoduše uchopitelné (- ideálně tedy nějaký pěkně spojitý matematický formalizmus, co se dá dobře vizualizovat)
- diskuze variant co používalo Die by the Sword:
    - Klávesnice - zastaralý koncept, moderní hráč počítá že k něčemu využije myš; stejně to tehdy byl jenom slabý odvar z toho co uměla myš
    - Myš - ze všech možností zvládá největší podmnožinu pohybů; vyžaduje cvik, ale umožňuje kreativitu a flexibilní reakce v nezvyklé situaci
    - Nahrávky - na první pohled pěkná myšlenka, v praxi však neumožňují flexibilně reagovat na specifickou situaci (hráč je schopen si vést okamžitý myšlenkový přehled o příliš malém množství nahrávek)

- výsledek diskuze: 
    - pro hráče chceme ovládání myší
        - jinou možnost než něco s myší, nebo něco s klávesnicí stejně nemáme
        - myš jediná má potenciál umožnit jemnou a flexibilní kontrolu (klávesnice má konečný počet kláves - konečný počet stavů)
    - interně implementujeme i přehrávání nahrávek - bude se hodit pro AI nepřítele (dává nám přímou, dobře uchopitelnou kontrolu nad jeho chováním a tedy obtížností hry) (jednoduchý systém, vhodný pro to v budoucnu sloužit jako základ, proti kterému budeme testovat sofistikovanější AI)
    - nějaké kreativnější nápady než s čím už přišlo DbtS necháme na později. teď chceme mít něco, co už víme, že je základně použitelné, abysme vůči tomu později nové výstřelky mohli porovnávat

- jakým formalizmem mapovat polohu myši na polohu meče:
    - nevíme jak přesně to na pozadí dělá DbtS
    - jednoduchý, dobře vizualizovatelný formalismus.. střílet paprsek z kamery z místa kurzoru a protnout ho s nějakým útvarem 
    - po dlouhé úvaze jsme se rozhodli, že pěkný útvar k protínání je kulová plocha
    - ...

\subsection{Jak bude vypadat výsledek}


\section{Shrnutí}
