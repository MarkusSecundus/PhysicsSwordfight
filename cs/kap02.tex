\chapter{Stanovení cíle}

V této kapitole se zamyslíme a stanovíme cíle, o něž tato práce bude usilovat. Následně vybereme nástroje, které nám k dosažení těchto cílů poslouží.  


\section{Jaký chceme výsledek}
\begin{itemize}
    \item \textbf{Zamyšlení}:
        \begin{itemize}
            \item neodsžitelným ideálem, ke kterému se snažíme rámcově směřovat, je zachycení zážitku, co šerm poskytuje v reálném světě
            \item kromě volného mávání mečem, ostatní aspekty co k boji v reálném světě patří, hry zvládají zachytit docela v pohodě -> ovládání meče poskytuje největší prostor k prozkoumání
            \item chceme obecný systém, co nebude potřebovat specifické úpravy pokaždé, když vytvoříme nový typ nepřítele, a co půjde použít k interakci s herním prostředím
            \item nechceme pro meč ručně vytvářet žádné animace (nemáme na to prostředky + to není programátorsky zajímavá činnost)
            \item meč by se měl tvářit v rámci možností rozumně fyzikálně věrně -> moderní herní enginy bývají vybaveny pokročilým fyzikálním subsystémem -> mohlo by být fajn nějakej takovej využít
        \end{itemize}
    
    \item \textbf{z pohledu backendu by meč měl zvládat plně volný pohyb na všech 6 stupních volnosti a měl by to být objekt plně ovládaný standardním fyzikálním subsystémem}
        \begin{itemize}
            \item zkusíme znovu to, co Die by the Sword, avšak tentokrát vytvoříme otevřenou platformu, na které bude moci jakýkoliv programátor z komunity experimentovat s metodami ovládání (nedoladěné ovládání (a nemožnost ho spravit komunitou) byl hlavní kámen úrazu DbtS)
            \item pokud se nám osobně nepodaří najít opravdu dobré a zábavné ovládání, někomu jinému z komunity se to povést může. i kdyby ne, na hru vždycky půjde napasovat VR ovladač ;)
        \end{itemize}
    \item \textbf{frontend - ovládání pro klávesnici a myš}
        \begin{itemize}
            \item jediná možnost, která zaručí přístupnost pro všechny
            \item mnoho prostoru k experimentálnímu průzkumu
        \end{itemize}
\end{itemize}

\section{Jaké nástroje použijeme}

\begin{itemize}
    \item chceme, aby výsledek naší práce byl co nejvíc přístupný pro náhodné lidi, co by ho chtěli využít k navazujícím experimentům či vlastnímu obohacení
    \item \textbf{použijeme herní engine Unity}
        \begin{itemize}
            \item je mature
            \item nejrozšířenější (viz \ref{unityEngineIntroSection}) -> největší šance že se najde jiný vývojář, co pro naší práci najde využití
            \item má spoustu standardních knihoven a subsystémů, co nám usnadní práci a zajistí rozumnou míru kompatibility s prací ostatních lidí
        \end{itemize}
    \item na všech místech práce chceme používat co nejvíc rozšířené a standardní systémy a knihovny (-> kompatibilita s prací ostatních lidí)
    \item myslet na dobré OOP practices a štábní kulturu
\end{itemize}

