%%% Fiktivní kapitola s ukázkami citací

\chapter{Odkazy na literaturu}

Odkazy na literaturu vytváříme nejlépe pomocí příkazů
\verb|\citet|, \verb|\citep| atp.
(viz {\LaTeX}ový balíček \textsf{natbib}) a~následného použití
Bib{\TeX}u. V~matematickém textu obvykle odkazujeme stylem \uv{Jméno
autora/autorů (rok vydání)}, resp. \uv{Jméno autora/autorů [číslo
odkazu]}. V~českém/slovenském textu je potřeba se navíc vypořádat
s~nutností skloňovat jméno autora, respektive přechylovat jméno
autorky. Je potřeba mít na paměti, že standardní příkazy
\verb|\citet|, \verb|\citep|
produkují referenci se jménem autora/autorů v~prvním pádě a~jména
autorek jsou nepřechýlena.

Pokud nepoužíváme bib\TeX{}, řídíme se normou ISO 690 a zvyklostmi
oboru.

Jména časopisů lze uvádět zkráceně, ale pouze v~kodifikované podobě.