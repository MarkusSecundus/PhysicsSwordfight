\chapter{Implementace}

\section{Implementace základních komponent}
\begin{itemize}
  \item \textbf{Šermíř}
    \begin{itemize}
      \item rigidbody (nekinematické)
    \end{itemize}
  \item \textbf{Meč}
    \begin{itemize}
      \item rigidbody (nekinematické)
      \item 
    \end{itemize}
\end{itemize}

\subsection{Moduly ovládání meče} \label{modesOfSwordMovementSubsection}
\begin{itemize}
  \item Co je modul z hlediska objektováho návrhu už čtenář ví, tady jenom popíšeme implementaci modulu pro sekání a modulu pro blokování
  \item \textbf{Abstrakce} 
    \begin{itemize}
      \item SwordMovement.Submodule
        \begin{itemize}
          \item samostatná abstraktní třída, nedědí z Unity.Object
          \item má metody jako OnUpdate, OnActivated, OnDeactivated apod. - za jejich volání nese zodpovědnost SwordMovement
          \item může volat metody na SwordMovementu (je do ní injectnuta instance rozrhraní ISwordMovement)
        \end{itemize}
      \item ISwordMovement
        \begin{itemize}
          \item Jednoduché rozhraní, exposuje jenom to, co submodul může potřebovat
          \item šikovné ho mít pro účely mockování v unit testech a taky protože takhle jsme schopní mezi SwordMovement a jeho komponenty nacpat dekorátor (kterej se tváří jako jediná aktivní komponenta, původní komponenty si ukradne pro sebe, sám se stará o správu toho, která je aktivní, a místo oficiálního SwordMovementu do nich injectne sebe) - takhle se dá např. elegantně nahrávat rozkazy pro pohyb meče (využijeme v \ref{knightEnemySubsection})
        \end{itemize}
      \item Správu toho která komponenta je aktivní jsme enkapsulovali do třídy ScriptSubmodulesContainer -> dekorátor mezi SwordMovementem a jeho Submodulama se dá napsat celkem bez námahy
      \item Nastavení submodulů z editoru - list párů klávesa-submodul ; submodule picker má dropdown, kde můžeme vybírat ze všech typů dědících ze SwordMovement.Submodule a pro vybraný typ se pak zobrazí defaultní picker - neskonalé díky: \href{https://github.com/lordconstant/SubclassPropertyDrawer}{lordconstant}
    \end{itemize}
  \item \textbf{Metoda SwordMovement.MoveSword()}
    \begin{itemize}
      \item vstupy: anchorPoint (poloha meče) + swordDirection (směr kam je namířený) + upDirection (směr nahoru - podle něj se počítá natočení kolem vlastní osy) + holdingForce (faktor síly držení)
      \item podle směru namíření a směru nahoru se nastaví cílová rotace mečového jointu
        \begin{itemize}
          \item stačí podle nich spočítat cílovou rotaci a tu nastavit jako targetRotation ConfigurableJointu (spočítat jí je ale netriviální problém - viz \ref{howToSetJointsTargetRotationSection})
        \end{itemize}
      \item nastavení polohy meče podle vyžádané cílové polohy...
        \begin{itemize}
          \item ConfigurableJoint podporuje stejně jako targetRotation i targetPosition - ta se ale při testování zdála nestabilní (čas od času meč odskočil 10 metrů vedle)
          \item ConfigurableJoint má property connectedAnchor - pozice místa kam je předmět jointem ukotvený, relativní vůči rigidbody ke kterému je joint ukotvený - jeho změnou můžeme efektivně mečem pohybovat
          \item potřeba nastavit Joint.autoConfigureConnectedAnchor = false
          \item chceme plynulý pohyb konzistentní rychlostí VS Joint.connectedAnchor se dá jenom natvrdo nastavit, neexistuje žádný targetConnectedAnchor 
          \item DOTween nejde použít (cílová pozice connectedAnchoru se potenciálně mění každý snímek - DOTween je stavěný na to, že cíl je známý od začátku a nemění se)
          \item řešení - vést si prostě target hodnotu a každý snímek connectedAnchor posunout o (target - currentConnectedAnchor)*deltaTime*speedFactor/*nastavitelný z editoru*/
          \item celou tuhle logiku jsme enkapsulovali do třídy RetargetableInterpolator - ta běží na pozadí jako korutina a jenom nastavujeme její target hodnotu
        \end{itemize}
      \item faktor síly držení
        \begin{itemize}
          \item důvod: někdy je správně defaultní síla, jakou hráč meč drží, někdy (modul blokování) ale např. potřebujeme, aby byl meč drženej tak silně, že rána druhého meče ho nedokáže srazit dolů
          \item na začátku si uložíme původní Joint.slerpDrive.positionSpring - nastavení faktoru síly držení znamená kolikanásobek téhle původní hodnoty se má začít používat
          \item plynulá změna - interpolujeme stejně jako Joint.connectedAnchor
          \item volitelný parametr - když není, bere se to jako by byla dodána hodnota 1 (-> 1-násobek původní síly)
        \end{itemize}
    \end{itemize}
  \item \textbf{Modul pro sekání:} 
    \begin{itemize}
      \item velmi jednoduchý
      \item má z editoru nastavitelný IRayIntersectable - prostě jenom vezme paprsek ze vstupu a intersectne ho
      \item když průnik není, nic neudělá
      \item když průnik je, nastaví hlášený střed koule jako handlePoint, lookDirection je (intersection.Value - intersection.Center)
      \item upVector je lookDirection.Cross(lastLookDirection) 
    \end{itemize}
  \item \textbf{Modul pro blokování:} 
    \begin{itemize}
      \item 
    \end{itemize}
\end{itemize}


\subsection{Animace šermíře}
\begin{itemize}
  \item 
\end{itemize}


\subsection{Počítání zranění}
\begin{itemize}
  \item 
\end{itemize}


\section{Stabilita simulace}
\begin{itemize}
  \item 
\end{itemize}


\subsection{Jak nastavit cílovou rotaci jointu} \label{howToSetJointsTargetRotationSection}
\begin{itemize}
  \item cílová rotace configurable jointu se udává v souřadnicovém systému, co je relativní vůči jointu
\end{itemize}

\subsection{Ladění parametrů} \label{swordParameterTweaksSection}
\begin{itemize}
  \item 
\end{itemize}

\subsection{Kolize mečů} \label{swordCollisionsSection}
\begin{itemize}
  \item 
\end{itemize}


\section{Demo hra}
\begin{itemize}
  \item Pro testování a demonstraci funkčnosti našeho konceptu jsme vytvořili ukázkovou hru
  \item herní svět - aréna + okolní les
  \item PC - šermíř, může chodit po světě a používat libovolně meč
  \item v aréně lze bojovat s jednoduchými NPC nepřáteli
  \item všechny modely a textury byly vytvořené autorem této práce a vztahuje se na ně MIT licence jako na všechno ostatní
\end{itemize}

\subsection{Herní svět}
\begin{itemize}
  \item les jsou jenom pěkné kulisy
  \item V aréně jsou sloupy s čudlíkama, po jejichž stisknutí mečem se uvnitř arény spawne nepřítel
  \item několik druhů nepřátel: trénovací panák, mečotoč, rytíř (podrobně rozebereme v jejich vlastních sekcích)
\end{itemize}

\subsection{GUI}
\begin{itemize}
  \item 
\end{itemize}

\subsection{Statičtí nepřátelé}
\begin{itemize}
  \item 
\end{itemize}


\subsection{Rytíř} \label{knightEnemySubsection}
\begin{itemize}
  \item 
\end{itemize}


\section{Shrnutí}
