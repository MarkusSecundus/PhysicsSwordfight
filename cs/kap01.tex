\chapter{Základy}
V této kapitole uvedeme čtenáře do problematiky boje s chladnou zbraní - velmi stručně nastíníme jeho vývoj a metodiku v reálném světě a následně do hloubky rozebereme mechaniky, skrze které ho adaptuje svět akčních videoher, a problémy, které při tom musí řešit.


\section{Chladná zbraň v reálném světě}
V této sekci čtenáře velmi stručně seznámíme s historickým pozadím a základními charakteristikami, které provázejí boj s chladnou zbraní v reálném světě.

\subsection{Průlet historií}

Není nadsázkou když prohlásíme, že chladná zbraň je koncept starý jako lidstvo samo. Archelogické nálezy nasvědčují, že již vzdálený předek moderního člověka Australopithecus používal úderný předmět (pravděpodobně kus dřeva či kost ulovené antilopy) jako \textbf{nástroj k lovu} paviánů \cite{AustralopithecusWeapon}. \textbf{Výhody}, které mu to mohlo přinést, jsou zjevné: zatímco kořist měla možnosti obrany omezené tím, kam dosáhly její vlastní zuby a drápy, lovec si ji mohl držet v bezpečné vzdálenosti a vystavovat vlastní tělo o poznání menšímu nebezpečí odvety.

V průběhu svého vývoje lidský rod zdokonaloval i své zbraně, což přineslo mnoho dalších implikací - ostrý kamenný hrot člověku umožnil zasazovat ránu spolehlivě, a hlubší, než by kdy umožnilo pouhé jeho tělo. Spolehlivost zbraně umožnila lepší \textbf{koordinaci mezi lovci}. Sofistikovaná koordinace člověku otevřela cestu k lovu větších a silnějších druhů zvěře. Nakonec se člověk propracoval na samý vrchol potravního řetězce a zbraň, nástroj lovu, sloužila stále více pro vzájemný \textbf{boj mezi lidmi}. 

Zrod civilizací vznesl na zbraň nové požadavky. Vysoká koncentrace lidí na jednom místě vedla přirozeně ke specializaci profesí, mezi jinými i vojenské. \textbf{Profesionální armáda} čítající tisíce lidí umožňovala (a vyžadovala) do té doby nevídanou míru koordinace - ideálem v takové situaci se ukázalo disponovat širokou škálou zbraní vysoce specializovaných, chladných i střelných, jež mohly být použity ve vzájemné synergii, doplněné válečnými stroji a vhodně vycvičenými zvířaty\footnote{kůň jak známo hrál přední roli}. Profesionální voják měl čas a motivaci svůj typ zbraně pochopit do hloubky, stejně tak vysoce náročná si mohla dovolit být i její výroba a údržba, za níž nesli zodpovědnost rovněž profesionálové ve svém oboru. 
Na druhé straně tu však byl \textbf{běžný obyvatel}, který nepatřil k pravidelnému vojenskému jádru, ale byla-li říše pod útokem, považovalo se za samozřejmost, že pozvedne zbraň na její obranu. Zbraň pro takového člověka vyžadovala především jednoduchost - jak na výrobu, tak na údržbu a použití, zkrátka aby bylo možné v časové tísni krizové situace dostat co nejrychleji co nejvíce odvedenců do bojeschopného stavu a udržet je v něm.

Mocenské soupeření států ústí ve zběsilý a nikdy nekončící \textbf{závod ve vývoji účinnějších zbraní a metod jejich použití}. V jednom okamžiku dominuje Blízkému východu vozataj, v následujícím jej poráží Makedonská falanga, nad ní předvede svou superioritu Římský systém manipulů, ten vzápětí Římané sami prohlašují za zastaralý, avšak o půl tisíciletí později stejně jejich říši rozvrací hordy Hunských jezdců... chladná zbraň, ať už v rukou jezdce či pěšího vojáka, zůstává po většinu dějin dominantním prvkem na bojišti, se střelnými zbraněmi hrajícími významnou podpůrnou roli. Až s rozmachem zbraní palných se tato dynamika obrací a velmi pozvolně se dobíráme k modernímu vojenství. V současné době se chladná zbraň považuje převážně za překonanou - uplatnění pro ní stále existuje např. v kontextu pořádkových složek, avšak i pro ty plní úlohu spíše doplňkovou. Mimo oficiální kruhy pořád hraje nezanedbatelnou roli, avšak to je především pro její \textbf{triviální dostupnost} v porovnání s palnou zbraní.

Tradice chladné zbraně však stále žije v civilních komunitách dedikovaných zachování historie a kulturního odkazu. Mezi významné patří japonské umění \textbf{Kendó} ("cesta meče") vycházející ze samurajské tradice, či evropská komunita \textbf{historického šermu}, jež vychází z dochovaného učení středověkých mistrů. Rovněž je zde moderní \textbf{sportovní šerm}, jež přímo navazuje na šermířskou tradici ranného novověku. V posledních letech tyto vlivy více pronikají i do běžných volnočasových aktivit - ve Střední Evropě například je stále oblíbenější fenomén \textbf{LARP}\footnote{\ac{LARP}}, jehož jedna z podob - tzv. "dřevárna" - znamená hromadnou akci, na které se desítky až stovky lidí v tématických kostýmech a vyzbrojení zpravidla dřevěnými, molitanem měkčenými replikami zbraní, střetnou v bitvě. Pravidla boje přirozeně musí zaručit bezpečí účastníků, avšak stále při zachování autentického zážitku z boje.  

Vzhledem k charakteru a rozšířenosti výše zmíněných fenoménů lze očekávat, že jejich zastánci mají v nemalé míře zastoupení i ve videoherní komunitě. Přirozeně takoví lidé mohou mít zájem o hru, jež jim umožní jejich oblíbenou činnost napodobit, ale bez mnohých nepohodlí, jež ji provázejí v reálném světě. Pro hráče, který s bojem v reálném světě žádnou zkušenost nemá, může zase videohra být cestou, jak tuto mezeru ve svých zkušenostech částečně zaplnit a získat například hlubší porozumění k historii.

\subsection{Průběh boje}
Nyní si představíme typický průběh boje vedeného s chladnými zbraněmi, a vypíchneme na něm znaky, které by videohra měla dokázat vystihnout, pokud se pokouší o realistický bojový systém.
\bigbreak

Cílem souboje je v nejjednodušším případě protivníka \textbf{zabít, vážně zranit či ho jinak vyřadit z bojeschopného stavu}, čehož může být dosaženo libovolnými prostředky, a ideálně zabránit, aby při tom stejný úděl potkal i vás. Není vyloučeno vítězství vyčerpáním protivníka či využití okolních předmětů a znalosti prostředí (např. lstivým vlákáním protivníka do bažiny).

Tato absolutní svoboda však může být omezena morálními zásadami účastníků\cite{HistoryOfSurrender} či předem dohodnutými pravidly souboje. Velmi zde záleží na \textbf{okolnostech}, při kterých k boji dochází:
\begin{itemize}
    \item Např. při \textbf{obraně domácnosti} před vetřelcem platí popis výše
    \item \textbf{Voják v bitvě} koná v mezích, jež mu stanovuje vojenská disciplína a rozkazy velitele
    \item \textbf{Účastník turnaje} přirozeně nesmí opustit vyhrazené bojiště či využívat pomoci z publika
    \item Ve \textbf{cvičném souboji} si účastníci logicky dávají pozor, aby jeden druhého zbytečně nezabil či nezmrzačil
    \item V duelu na obranu cti se účastníci mohli \textbf{předem dohodnout}, že nechtějí jeden druhého zabít, a např. bojovat "na první krev"
\end{itemize}
Mezi znaky zkušeného šermíře patří schopnost být si vědom těchto mantinelů a v jejich rámci jednat kreativním a účinným způsobem.

Extrémní případ těchto omezení vidíme v moderních šermířských komunitách. U těch najdeme velmi \textbf{striktní pravidla} a vysoké nároky na bojovou výstroj s cílem zamezit jakýmkoliv vážným zraněním a ztrátám na životech, jelikož ty přirozeně nejsou cílem jejich snažení. 

Avšak pokud nedochází k žádným reálným zraněním, \textbf{jak lze rozhodnout, kdo vyhrál?} Zjevnou cestou je například systém zásahových zón a počítání životů, které se bojovníkovi v závislosti na zasažené zóně odečítají. Takový systém však nezaručuje realistický dojem z boje. Mezi plody snah o realističtější zážitek patří např. v české LARPové scéně oblíbený šatrh\cite{Satrh}, založený na důvěře v účastníky, že podle způsobu, jakým byli zasaženi, odhadnou a zahrají realistickou reakci, či buhurt, kde zkrátka vítězí ten, kdo nepadne vyčerpáním. Je tedy vidět, že i komunity, jejichž nejvyšší prioritou při boji je záruka bezpečí, jsou ochotné vyvinout značnou snahu, aby byl zážitek z boje realistický.

\subsection{Boj jeden na jednoho}
Jak tedy probíhá duel mezi dvěma šermíři? Hloubkový rozbor zohledňující historické i moderní techniky, rozdílné úrovně zkušenosti, zdatnosti a motivace obou zápasníků, veškeré obvyklé zbraně, kterými by mohli být vybaveni, prostředí a vnější okolnosti, které v souboji mohou hrát podstatnou roli, není možné provést v rozsahu bakalářské práce, natož jedné její kapitoly. Pro ten tedy čtenáře odkáži na specializovanou literaturu \cite{KunstDesFechtens} \cite{FightingWithTheGermanLongsword} \cite{ModernHEMA}.  

Zde si pouze nastíníme pár základních faktorů, které při souboji hrají důležitou roli.

\textbf{Obratnost zbraně} - schopnost rychle měnit její trajektorii, reagovat na změny situace (např. vyblokovat nečekaný úder) či změny situace sám vyvolat (např. na poslední chvíli zastavit fingovaný úder a stočit čepel na jiné místo, které protivník nechal nechráněné). Mezi faktory, kterými je determinována, patří velikost a hmotnost zbraně, její tvar, těžiště a způsob úchopu\footnote{Čím jsou ruce dál od sebe, tím silnější je princip páky.}; v neposlední řadě také síla a tělesné rozpoložení šermíře.

\textbf{Dosah, pohyb, terén} - V okamžiku, kdy vám protivník není schopen ublížit a vy jemu ano, můžete si dovolit být libovolně agresivní\footnote{Díky tomu dlouhé zbraně jako kopí mohou být velmi účinné i v rukou nezkušeného odvedence.}. Je-li váš protivník v takovéto výhodě, jedinou možností je se k němu přiblížit - pak se dostanete do výhody vy, jelikož zbraně s dlouhým dosahem zpravidla trpí na nízkou obratnost při boji zblízka\footnote{Nedejte protivníkovi čas tasit poboční zbraň.}. Vaše možnosti, jak se k němu přiblížit, přirozeně velmi závisí na terénu\footnote{Když stojíte pod hradbami a protivník bodá kopím z ochozu nad vámi, jste v pěkné kaši.}. 

Máte-li oba stejný dosah, pohyb se však nestává o nic méně důležitým. Správná technika pohybu pomáhá šermíři držet pod kontrolou jeho těžiště a nenechat se rozhodit, když klopýtne, či na něj protivník agresivně naběhne; případně mu umožní bleskově se přiblížit a využít nechráněnou oblast, kterou mu protivník nabídl svým nemotorným pohybem. 

\textbf{Zbroj}, kterou má protivník na sobě, může citelně omezit vaše možnosti co do způsobů, jak mu uštědřit zranění. Je-li například v plné plátové zbroji pozdního středověku, pokoušet se do ní sekat mečem je prakticky zbytečné. Vaše možnosti se citelně zúží - buď se můžete pokusit bodnout do některé ze skulin ve zbroji, nebo meč vyměnit za palcát či válečné kladivo\footnote{Případně lze meč chytit za čepel a záštitou zasuplovat hlavici kladiva - viz \cite{FightingWithTheGermanLongsword}} a protivníkovi tupými údery zlámat kosti. Protivník si vaší situace bude vědom, a o to jednodušší pro něj bude předvídat vaše akce.

Znakem každého dobrého šermíře je schopnost vcítit se do pozice svého protivníka a \textbf{odhadovat jeho budoucí akce} dříve, než se stanou. Nezkušený šermíř bude každý svůj úder provázet výrazným nápřahem, pro jeho protivníka pak není obtížné úder vyblokovat a přitom se rovnou k protivníkovi přiblížit a seknout po místě, které ví, že při úderu protivník nechal nechráněné. Zkušený šermíř pak ani nepotřebuje vidět výrazný nápřah, naučí se protivníkovu akci podvědomě vycítit z drobných náznaků (pohybů svalů, očí,...). Samozřejmě je možné předvádět údery fingovaně a čekat, že se při protiútoku otevře protivník vám. Pak záleží na jeho úsudku, zda úmysl prokoukne a provede fingovaný protiútok - stupňům falše se meze nekladou. 

\subsection{Hromadný boj a boj proti přesile}
Souboj jeden na jednoho však může vypadat jako téměř laboratorní podmínky vedle chaotických situací, v rámci kterých v historii běžně k boji docházelo.   

Často se střetávaly větší, více či méně \textbf{organizované skupiny}, případně celé armády čítající mnoho takových pevně ustanovených jednotek. V takové situaci je na místě, aby bojovníkova individualita ustoupila do pozadí - rozhodujícím faktorem je až na výjimky taktické jednání a sladěnost celku. 

Jedním ze základních požadavků, které jsou na vojáka kladeny, je schopnost \textbf{držení formace}. Stojíte-li jako semknutá řada, štít na štít, meče připravené opětovat jakýkoliv úder, který by protivník zkusil na kamaráda vedle, je mnohem těžší prorazit vašimi řadami a dostat se k lučištníkům, kteří nepříteli způsobují těžké ztráty. 

Při skupinové šarvátce hraje zpravidla rozhodující roli taktická \textbf{koordinace} spolubojovníků. Nepřítel, který vám překvapivě vpadl do zad, je mnohem nepříjemnější záležitost, než nepřítel, který postává opodál a čeká, až na něj přijde řada.  

Pro dosažení koordinace je však nutná \textbf{komunikace}. Dohodnout se při šarvátce na postupu, v omezeném čase a aniž by plán byl zachycen protivníkem, vskutku není triviální problém. V kontextu větší bitvy pak důležitost i obtížnost komunikace nabývá zcela nového rozměru a jde o těžký problém, který až do vynálezu moderní vysílačky nebyl uspokojivě vyřešen. Vojenská taktika si vyžaduje flexibilní reakce na změny situace na bojišti, informace musí proudit mezi vojenskými jednotkami a velitelstvím, které jsou často netriviálně fyzicky vzdálené. Jednou možností, jak informaci předat, je vyslání posla - ten je schopen nést detailní informaci, je však zranitelný a pomalý - než dorazí, může již jeho informace být neaktuální. Protipólem pak je broadcast předem dohodnutých signálů, napříkald prostřednictvím trubače - tato cesta je rychlá a v určitém doslechovém okruhu vcelku spolehlivá, avšak signály musí být předem dohodnuté a variant nemůže být mnoho, metoda tedy postrádá flexibilitu a detailnost informace. Jak vidno, jde o zásadní součást vedení bitvy, přesto však vidíme mnoho videoher, které ji zcela zanedbávají.
\bigbreak
Dojde-li k boji \textbf{jediného bojovníka proti přesile}, vše výše uvedené stále platí, akorát očividně kolosální výhody plynoucí z koordinovaného postupu jsou dopřány pouze jedné ze stran. I pro velmi zkušeného šermíře bojujícího proti několika nezkušeným, často jediná šance, jak z takové situace vyváznout, je protivníky rozdělit (kreativním použitím pohybu a znalosti terénu) a zdolat jednoho po druhém rychle, než mu ostatní přijdou na pomoc.

\subsection*{Shrnutí}
Seznámili jsme čtenáře s historickým pozadím a základními prvky, které charakterizují boj s chladnou zbraní v reálném světě. Nyní by měl být připraven pokročit dál a provést srovnání s přístupem, jakým tuto problematiku adaptuje svět videoher.

\clearpage
%------------------------------------------------------------------------------------------------------------------------------------------------------------------------------------------------------------------------------------------------------------%
 % xxxxxxxxxxxxxxxxxxxxxxxxxxxxxxxxxxxxxxxxxxxxxxxxxxxxxxxxxxxxxxxxxxxxxxxxxxxxxxxxxxxxxxxxxxxxxxxxxxxxxxxxxxxxxxxxxxxxxxxxxxxxxxxxxxxxxxxxxxxxxxxxxxxxxxxxxxxxxxxxxxxxxxxxxxxxxxxxxxxxxxxxxxxxxxxxxxxxxxxxxxxxxxxxxxxxxxxxxxxxxxxxxxxxxxxxxxxxxxxxxxxxxxxx %
%------------------------------------------------------------------------------------------------------------------------------------------------------------------------------------------------------------------------------------------------------------%


\section{Chladná zbraň v akčních videohrách}

Nyní, když máme rámcovou představu, jak boj s chladnou zbraní vypadá v reálném světě, nastíníme si, jakým způsobem bývá typicky adaptován v akčních videohrách - jaké jeho aspekty se podařilo v tomto universu věrně zpodobnit, jaké byly ignorovány, a jaké interpretovány s řádnou dávkou umělecké kreativity.

\subsection{Obecný přehled}

Účinkujícími ve hře jsou \textbf{herní postavy}. Ty mohou být ovládané buď hráčem (takovou postavu nazveme \textbf{\acs{PC}}\footnote{\Acl{PC}}), nebo počítačovým algoritmem (tu označíme jako \textbf{\acs{NPC}}\footnote{\Acl{NPC}}), a lze je popsat nějakým jejich \textbf{vnitřním stavem}. Nyní si představíme základní veličiny, které jsou typickou součástí vnitřního stavu postavy ve všech akčních hrách bez ohledu na to, zda jde o hru zaměřenou na boj se střelnou zbraní, chladnou zbraní, či boj jakýkoliv jiný:

\begin{itemize}
    \item \textbf{\ac{HP}}... Číslo udávající zdravotní stav a životaschopnost postavy. \textbf{Pokud klesne na nulu, postava umírá.} K jeho snížení může dojít například úderem protivníkovy zbraně či pádem z výšky a takto odebraným bodům HP říkáme \textbf{\acs{dmg}}\footnote{\Acl{dmg}}. Hráči je stav jeho HP obvykle v reálném čase explicitně komunikován skrze ukazatel na obrazovce. V průběhu boje typicky nejsou doplňovány\footnote{Popřípadě prostředky, skrze které je lze doplnit, jsou vzácné.} a postava musí vystačit s tím, co má - jde tedy o zdroj vybízející k dlouhodobějšímu plánování.
    \item \textbf{Výdrž}... Volitelný avšak častý prvek, číslo reprezentující akutní schopnost postavy vynakládat fyzické úsilí. Spotřebováváno chvatným pohybem a prováděním akcí, v průběhu boje zpravidla dochází k jeho rychlému opakovanému spotřebování a regeneraci.
    \item \textbf{Inventář}... List dalších předmětů, které postava nese s sebou. Může jít o peníze, léčivé lektvary, munici, náhradní zbraně, magické svitky, mapy a cokoliv dalšího v závislosti na hře. 
\end{itemize}

V závislosti na vstupu, který získává od hráče či od řidícího algoritmu, se postava pohybuje po herním světě, interaguje s objekty herního světa a s obsahem svého inventáře a používá svou zbraň. Na použití zbraně se nyní zaměříme.

\textbf{Ovládání zbraně} bývá typicky omezené na stisknutí tlačítka - \textbf{Zaútoč!} - načež postava sama provede útok tak, jak uzná za vhodné - obvykle přehráním jedné ze seznamu animací, které pro danou kombinaci postavy a zbraně předem připravil autor hry.

Implikace, které takto jednoduché rozhraní přináší \textbf{pro hráče}, jsou následující:
\begin{itemize}
    \item \textbf{Intuitivnost} - Hráč, který poprvé spustí hru a zkusí náhodně mačkat tlačítka, rychle ovládání pochopí a začne být ve hře použitelný - žádná nutnost procházet úmorným tutoriálem.
    \item \textbf{Uniformní rozhraní pro všechny zbraně} - Tímto způsobem lze ovládat libovolnou zbraň, ať už jde o dvoumetrovou halapartnu či pěsti hráčovy postavy. Hráč tedy není přehlcen množstvím jednoúčelových herních systémů a z výhod, které získá svým zdokonalováním v použití bojového systému, může těžit po celou hru.
    \item \textbf{Nízká specifičnost} - Rozhraní hráči neumožňuje těžit ze silných stránek žádné konkrétní zbraně. Hra mu skrze bojový systém není schopna poskytnout vzdělávací lekci ohledně vlastností zbraně v reálném světě. Pokud mu tyto vlastnosti již jsou známé, může utrpět jeho imerze.
    \item \textbf{Limitovaná předvídatelnost} - Vzhledem k tomu, že akci, která bude vykonána, nevybírá hráč, ale jeho herní postava, vnáší se tím do hry prvek náhody, který má potenciál hru emocionálně okořenit.\footnote{Pokud však výběr akce probíhá nějakým deterministickým způsobem, hráč v něm brzy intuitivně vypozoruje určité vzory a naučí se akce své postavy do jisté míry předvídat a kalkulovat s nimi při plánování svého postupu - tím může bojový systém získat na hloubce.}
\end{itemize}

A co důsledky, které tento přístup přináší z pohledu \textbf{tvůrce hry}?:
\begin{itemize}
    \item \textbf{Předvídatelnost} - Tvůrce hry má přesnou kontrolu nad tím, jaké akce může jakákoliv postava potenciálně vykonat. Tím se zmenšuje prostor pro výskyt neošetřených okrajových případů.
    \item \textbf{Uniformní rozhraní pro všechny zbraně} - Herní logika, která umí pracovat s jednou zbraní, umí pracovat s každou zbraní. To může dopomoci k celkově čistému a kvalitnímu kódu a značně se tím snižuje námaha s přidáváním dalších zbraní (či jejich procedurální generací za běhu).
    \item \textbf{Nízká specifičnost} - Není možné uspokojivě ztvárnit veškerá zajímavá specifika konkrétní zbraně. Všechny zbraně se musejí chovat do jisté míry podobně. Často je tak zbraň smrštěna do pouhé tabulky statů\footnote{Povrchní veličiny typu dosah, rychlost úderu, míra způsobeného zranění}.
\end{itemize}

Vidíme, že takovéto pojetí boje se zbraní má mnohá pozitiva, stále je však očividné, že \textbf{jde o systém jednoduchý, který sám o sobě nemůže sloužit jako nosná herní mechanika v akční videohře} - oblasti, kde cílová hráčská skupina očekává určitou míru komplexity a hloubky.

Hloubka bývá tomuto systému zpravidla dodávána výběrem mezi několika různými typy útoku (každý efektivní v jiné situaci), akcí pro blokování nepřátelských útoků, důrazem na časování (krátkodobé omráčení nepřítele, navazování úderů, odměna za vyblokování nepřítele v přesně správném okamžiku,...) a \textbf{silným propojením s dalšími mechanikami}, jakými je v první řadě pohyb, dále různé lektvary, magie apod. v závislosti na hře.

Toto si nyní názorně předvedeme na příkladě komerčně úspěšné hry z nedávných let.

\subsection{Zaklínač 3 - vzorový příklad}
Zaklínač 3: Divoký Hon \cite{Wiedzmin3} je akční \acs{RPG} hra, která byla vydána r. 2015 polským studiem CD Projekt. Mezi její hlavní důrazy patří \textbf{otevřený fantasy svět}, propracovaný \textbf{nelineární příběh}\footnote{Hra vychází z knižní série Zaklínač Andrzeje Sapkowského \cite{SapkowskiZaklinac}} a v neposlední řadě také dynamický \textbf{bojový systém}.

Hráč se zde ujímá role Geralta z Rivie - zaklínače, \textbf{profesionálního lovce monster}. Ten je kromě svých dvou mečů\footnote{stříbrného pro boj s nestvůrami a železného proti lidem} vybaven flexibilními možnostmi pohybu, jednoduchými kouzly (tzv. Zaklínačská Znamení) a škálou ručně vařených lektvarů. Na své cestě bojuje s lidmi, duchy, upíry, bazilišky i nepopsatelnými obludami z nočních můr.
Jak tedy vypadají mechaniky, skrze které bylo toto vše umožněno?:
\bigbreak

Základem jsou \textbf{dva typy úderů - rychlý a silný}. Jak vyplývá z názvu, rychlý úder je rychlejší, oproti tomu animace silného úderu trvá mírně delší čas, avšak zranění způsobené protivníkovi je také adekvátně zvýšené. K čemu je to dobré? Pro mnoho protivníků platí, že v okamžiku, kdy jim je učiněno nezanedbatelné zranění, přeruší akci, kterou se chystali provádět, a na zlomek času zakolísají. S rychlým úderem je hráč schopen toto zakolísání vyvolat častěji - s dobrým časováním se lze například dostat do cyklu, kdy lehkooděného protivníka hráč udolá jedním úderem za druhým aniž ten se zmohl udělat cokoliv proti vám. Toto stejné zakolísání však může potkat i hráče. Hráč se tedy neustále nachází v časovém tlaku, který ho vybízí kalkulovat - např. zda stihne dva rychlé údery, či raději jen jeden silný, než ho nějaký nepřítel přeruší. Dalším důvodem proč používat silný útok jsou těžkoodění nepřátelé, kterým rychlý útok udílí zranění velmi redukované. 

Drobným doplňkem je akce \textbf{blokování}. Bojujete-li proti lidem, držení tlačítka blokování vás činí prakticky imunním proti běžným útokům mířícím na vás zepředu, za tu velmi drobnou cenu, že sám nemůžete útočit a pohybovat se po bojišti lze jen velmi pomalu. Zajímavější je \textbf{perfektní blokování} - pokud zahájíte blokování přesně v okamžiku, kdy na vás protivník vede úder, provedete odvetný protiúder, který protivníka na pár sekund omráčí. Tím je do boje vnesen další prvek časování. Rovněž je třeba dávat pozor na nepřátele, kteří blokují proti vám.

Proti chaotickým útokům vedeným změtí zubů a drápů nestvůry však blokování mečem ztrácí smysl. Tehdy přichází na řadu možnosti úskoků a kotoulů, které nabízí \textbf{systém pohybu}. Kromě nich samozřejmě hra nabízí i klasickou chůzi a sprint. Úskok je velmi drobný posun do strany, který nespotřebovává výdrž, pouze má interní cooldown (několik desetin sekundy) zabraňující spamování. Cílem je umožnit hráči tak akorát uskočit z oblasti dopadu nepřítelovy zbraně. Pro zajištění, že tento záměr bude vždy naplněn, je postavě na malý zlomek sekundy poskytnuta celková nezranitelnost. Kotoul je pak rámcově obdobný s tím rozdílem, že spotřebovává výdrž a na oplátku ho lze použít k rychlému překonání velké vzdálenosti. Hra klade velký důraz na hromadné boje, kde hráč musí být schopen prioritizovat cíle a rychle využívat objevivších se zranitelností jednotlivých nepřátel. Vysoká mobilita, kterou kotoul poskytuje, je tedy více než nápomocná.  

Zaklínačská Znamení do boje přinášejí další vrstvu komplexity. Jde o \textbf{5 kouzel}, která spotřebovávají velkou porci výdrže a na oplátku zaklínači mocně pomohou - např. může jít o ochranné silové pole, poryv větru, který má šanci omráčit zasažené protivníky, či zmatení, které přiměje nepřítele bojovat na vaší straně.

Výčet Znamení je pevně stanoven, avšak klasicky pro žánr RPG, hra zahrnuje \textbf{prvky progrese}, pomocí kterých lze Znamení v průběhu hry vylepšovat a modifikovat. Stejné platí i pro boj s mečem - rychlý i silný úder mají každý své vlastní stromy dovedností, které jsou schopny zajímavými způsoby ovlivnit gameplay. 

Poslední částí zaklínačova repertoáru, kterou zmíníme, jsou lektvary a oleje. Jde o \textbf{konzumovatelné předměty}, které hráč v průběhu hry sám vyrábí, a po jejich konzumaci (požití lektvaru postavou / namazání oleje na meč) mu je na určitý časový limit poskytnut nějaký bonus (př. regenerace zdraví, bonusové zranění proti upírům apod.). Jde tedy o zdroje, se kterými hráč hospodaří v dlouhodobějším horizontu, ale použity v kritickém okamžiku mohou zvrátit výsledek boje. 

Esenciální součástí každé hry s bojovými prvky, jsou také \textbf{nepřátelé}. V této oblasti byl Zaklínač 3 k hráči velmi štědrý. Najdeme zde obyčejné lidské nepřátele (bandity, nepřátelské vojáky apod.), jejichž způsob boje sdílí mnoho prvků se zaklínačovým - např. jsou schopni blokovat jeho údery. Dále pak širokou paletu monster všech velikostí a tvarů, pozemní i schopné létat vzduchem, vybuchující, duchy zranitelné pouze specifickými cestami, tvory vládnoucí magií - každý nepřítel má své osobité silné a slabé stránky a vybízí hráče k jinému stylu hraní.

\bigbreak

Vidíme, že bojový systém Zaklínače 3 je zjevně postaven kolem kostry, kterou jsme si načrtli v předchozí kapitole, zároveň jde o systém mající značnou hloubku. Tato hloubka je získána z velké části důrazem na celkový hráčův přehled o situaci na bojišti, škálou osobitých nepřátel, důrazem na časování, a pak také propojením s dalšími systémy - pohybem, magií a vařením lektvarů. 

Hráči je poskytnut prostor k budování dovednosti a k důmyslnému taktizování jak v krátkodobém, tak v dlouhodobém horizontu. Celkové kvality bojového systému ostatně potvrzuje značný komerční úspěch, kterého se hře po vydání dostalo.

Mezi možné příčiny úspěchu hry můžeme zařadit také pozorování, že \textbf{hloubka systému je volitelná}. Rekreační hráč, který hraje převážně kvůli příběhu, si ji může dovolit víceméně ignorovat. Stačí hru nastavit na nízkou obtížnost a většinu nepřátel není těžké udolat pouze s povrchní znalostí základních mechanik. Není odvážné tvrdit, že to může mít pozitivní vliv na velikost cílového hráčského publika. 

\subsection{Chladná zbraň jako doplňkový prvek}

Chladnou zbraň najdeme také v mnoha hrách, ve kterých není stěžejním prvkem. Například ve většině her žánru \textbf{\acs{FPS}} najdeme kromě širokého výběru palných zbraní i nějaký nůž jako poslední zálohu když dojde munice, v žánru \textbf{stealth} často vidíme chladnou zbraň jako prostředek k tiché likvidaci nepřátel a takto bychom mohli pokračovat. Mnohdy taková zbraň odpovídá popisu z 1.2.1 - cílem není, aby chladná zbraň sama o sobě byla zajímavou mechanikou, naopak \textbf{tvoří pouze jeden dílek} ve skládačce herního systému a pro hráče musí být jednoduše uchopitelná. 

Přesto najdeme mnoho takových zbraní, které se staly \textbf{ikonickým symbolem} své hry - zmínit můžeme například hru DOOM \cite{DOOM1993} a motorovou pilu, či páčidlo ze série Half-Life \cite{HalfLife}.


\section{Hry s větší kontrolou}

Nyní víme, jakým způsobem je typicky chladná zbraň v akčních hrách ztvárňována, známe pozitiva, která tento způsob přináší, a rovněž některé triky, jak touto cestou dosáhnout poutavého bojového systému. Co nás ale čeká, kdybychom chtěli sejít z vyšlapané cesty a nabídnout hráči větší míru kontroly nad jeho zbraní?

Neprve se zamysleme, proč by hráč o větší kontrolu měl mít zájem:
\begin{itemize}
    \item fdsfds
\end{itemize}


\clearpage
(proč by hráč měl chtít větší kontrolu)
- je to poučné o reálném světě
- baví ho dělat blbosti, prostě chce mít kontrolu 
- (může mít trauma ze hry kde postava udělala blbost co nechtěl)?
- baví ho šermovat IRL

(proč by vývojář měl chtít větší kontrolu pro hráče)
- emergentní herní mechaniky
- interakce s env., velmi diegetické GUI

(Problémy co je při tom potřeba vyřešit)
- narativní disonance (pc má být zkušený šermíř)
- hratelnost (telegrafování nepřátelských úderů apod.)
- podchycení edge casů
- (hráči jsou zvyklí na click->action z FPS)?
- kompromisní ovládání s myší+klávesnicí/ovladačem


\subsection{Die by the Sword}

\clearpage
(stručné představení hry)

(popis bojového systému)

(jaké výhody z tohoto plynou)

(jaké výhody se nepodařilo získat)


\subsection{Mount\&Blade}

(stručné představení hry)

(popis bojového systému)

(jaké výhody z tohoto plynou)

(jaké výhody se nepodařilo získat)

\subsection{Kingdom Come: Deliverance}

(stručné představení hry)

(popis bojového systému)

(jaké výhody z tohoto plynou)

(jaké výhody se nepodařilo získat)


\section{Alternativní ovladače}

(představení - ovladače co mohou být pro náš účel ovládání meče víc vhodné - existují ovladače co se postupně dostávají do mainstreamu, co jsou state-of-the art ve svém obskurním niche oboru, i naprosto experimentální ovladače co jsou teprve předmětem výzkumu)

\subsection{Virtuální realita}

(jaké má výhody)

(příklad hry kde se s ním ovládá meč) (swords\&sorcery, gorn, beatsaber)

(nevýhody)
- disonance "meč narazil do něčeho a není tam kde ovladač"

(proč se jí nebudeme při výzkumu zabývat)
- pořád není mainstream
- (zjistit) uzavřenost platforem?


\subsection{Pohybové senzory ve smartphonu}

(výhody)
- dostupnost
- otevřenost platformy, programátorská flexibilita (hackovatelnost)

(nevýhody)
- fragmentace mobilního trhu, nekonzistence výstupu senzorů mezi různými mobily
- způsob připojení (ne každej desktop má bluetooth nebo wifi, kabel je nepraktickej; celkově mobily asi maj problém s nízkolatenčním přenosem po místní síti)
- najít nějakej článek

\subsection{Obskurní periferie}

3d myš
- extrémně drahá 
- stavěná pro 3d modelováni
- nemá 6 ale jenom 3 stupně volnosti


\section{Shrnutí}
