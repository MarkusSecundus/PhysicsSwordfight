\chapter{Úvod do problematiky}
V této kapitole uvedeme čtenáře do problematiky boje s chladnou zbraní - velmi stručně nastíníme jeho vývoj a metodiku v reálném světě a následně do hloubky rozebereme mechaniky, skrze které ho adaptuje svět akčních videoher, a problémy, které při tom musí řešit.


\section{Chladná zbraň v reálném světě}
V této sekci čtenáře velmi stručně seznámíme s historickým pozadím a základními charakteristikami, které provázejí boj s chladnou zbraní v reálném světě.

\subsection{Průlet historií}

Není nadsázkou když prohlásíme, že chladná zbraň je koncept starý jako lidstvo samo. Archelogické nálezy nasvědčují, že již vzdálený předek moderního člověka Australopithecus používal úderný předmět (pravděpodobně kus dřeva či kost ulovené antilopy) jako \textbf{nástroj k lovu} paviánů \cite{AustralopithecusWeapon}. \textbf{Výhody}, které mu to mohlo přinést, jsou zjevné: zatímco kořist měla možnosti obrany omezené tím, kam dosáhly její vlastní zuby a drápy, lovec si ji mohl držet v bezpečné vzdálenosti a vystavovat vlastní tělo o poznání menšímu nebezpečí odvety.

V průběhu svého vývoje lidský rod zdokonaloval i své zbraně, což přineslo mnoho dalších implikací - ostrý kamenný hrot člověku umožnil zasazovat ránu spolehlivě, a hlubší, než by kdy umožnilo pouhé jeho tělo. Spolehlivost zbraně umožnila lepší \textbf{koordinaci mezi lovci}. Sofistikovaná koordinace člověku otevřela cestu k lovu větších a silnějších druhů zvěře. Nakonec se člověk propracoval na samý vrchol potravního řetězce a zbraň, nástroj lovu, sloužila stále více pro vzájemný \textbf{boj mezi lidmi}. 

Zrod civilizací vznesl na zbraň nové požadavky. Vysoká koncentrace lidí na jednom místě vedla přirozeně ke specializaci profesí, mezi jinými i vojenské. \textbf{Profesionální armáda} čítající tisíce lidí umožňovala (a vyžadovala) do té doby nevídanou míru koordinace - ideálem v takové situaci se ukázalo disponovat širokou škálou zbraní vysoce specializovaných, chladných i střelných, jež mohly být použity ve vzájemné synergii, doplněné válečnými stroji a vhodně vycvičenými zvířaty\footnote{kůň jak známo hrál přední roli}. Profesionální voják měl čas a motivaci svůj typ zbraně pochopit do hloubky, stejně tak vysoce náročná si mohla dovolit být i její výroba a údržba, za níž nesli zodpovědnost rovněž profesionálové ve svém oboru. 
Na druhé straně tu však byl \textbf{běžný obyvatel}, který nepatřil k pravidelnému vojenskému jádru, ale byla-li říše pod útokem, považovalo se za samozřejmost, že pozvedne zbraň na její obranu. Zbraň pro takového člověka vyžadovala především jednoduchost - jak na výrobu, tak na údržbu a použití, zkrátka aby bylo možné v časové tísni krizové situace dostat co nejrychleji co nejvíce odvedenců do bojeschopného stavu a udržet je v něm.

Mocenské soupeření států ústí ve zběsilý a nikdy nekončící \textbf{závod ve vývoji účinnějších zbraní a metod jejich použití}. V jednom okamžiku dominuje Blízkému východu vozataj, v následujícím jej poráží Makedonská falanga, nad ní předvede svou superioritu Římský systém manipulů, ten vzápětí Římané sami prohlašují za zastaralý, avšak o půl tisíciletí později stejně jejich říši rozvrací hordy Hunských jezdců... chladná zbraň, ať už v rukou jezdce či pěšího vojáka, zůstává po většinu dějin dominantním prvkem na bojišti, se střelnými zbraněmi hrajícími významnou podpůrnou roli. Až s rozmachem zbraní palných se tato dynamika obrací a velmi pozvolně se dobíráme k modernímu vojenství. V současné době se chladná zbraň považuje převážně za překonanou - uplatnění pro ní stále existuje např. v kontextu pořádkových složek, avšak i pro ty plní úlohu spíše doplňkovou. Mimo oficiální kruhy pořád hraje nezanedbatelnou roli, avšak to je především pro její \textbf{triviální dostupnost} v porovnání s palnou zbraní.

Tradice chladné zbraně však stále žije v civilních komunitách dedikovaných zachování historie a kulturního odkazu. Mezi významné patří japonské umění \textbf{Kendó} ("cesta meče") vycházející ze samurajské tradice, či evropská komunita \textbf{historického šermu}, jež vychází z dochovaného učení středověkých mistrů. Rovněž je zde moderní \textbf{sportovní šerm}, jež přímo navazuje na šermířskou tradici ranného novověku. V posledních letech tyto vlivy více pronikají i do běžných volnočasových aktivit - ve Střední Evropě například je stále oblíbenější fenomén \textbf{LARP}\footnote{\ac{LARP}}, jehož jedna z podob - tzv. "dřevárna" - znamená hromadnou akci, na které se desítky až stovky lidí v tématických kostýmech a vyzbrojení zpravidla dřevěnými, molitanem měkčenými replikami zbraní, střetnou v bitvě. Pravidla boje přirozeně musí zaručit bezpečí účastníků, avšak stále při zachování autentického zážitku z boje.  

Vzhledem k charakteru a rozšířenosti výše zmíněných fenoménů lze očekávat, že jejich zastánci mají v nemalé míře zastoupení i ve videoherní komunitě. Přirozeně takoví lidé mohou mít zájem o hru, jež jim umožní jejich oblíbenou činnost napodobit, ale bez mnohých nepohodlí, jež ji provázejí v reálném světě. Pro hráče, který s bojem v reálném světě žádnou zkušenost nemá, může zase videohra být cestou, jak tuto mezeru ve svých zkušenostech částečně zaplnit a získat například hlubší porozumění k historii.

\subsection{Průběh boje}
Nyní si představíme typický průběh boje vedeného s chladnými zbraněmi, a vypíchneme na něm znaky, které by videohra měla dokázat vystihnout, pokud se pokouší o realistický bojový systém.
\bigbreak

Cílem souboje je v nejjednodušším případě protivníka \textbf{zabít, vážně zranit či ho jinak vyřadit z bojeschopného stavu}, čehož může být dosaženo libovolnými prostředky, a ideálně zabránit, aby při tom stejný úděl potkal i vás. Není vyloučeno vítězství vyčerpáním protivníka či využití okolních předmětů a znalosti prostředí (např. lstivým vlákáním protivníka do bažiny).

Tato absolutní svoboda však může být omezena morálními zásadami účastníků\cite{HistoryOfSurrender} či předem dohodnutými pravidly souboje. Velmi zde záleží na \textbf{okolnostech}, při kterých k boji dochází:
\begin{itemize}
    \item Např. při \textbf{obraně domácnosti} před vetřelcem platí popis výše
    \item \textbf{Voják v bitvě} koná v mezích, jež mu stanovuje vojenská disciplína a rozkazy velitele
    \item \textbf{Účastník turnaje} přirozeně nesmí opustit vyhrazené bojiště či využívat pomoci z publika
    \item Ve \textbf{cvičném souboji} si účastníci logicky dávají pozor, aby jeden druhého zbytečně nezabil či nezmrzačil
    \item V duelu na obranu cti se účastníci mohli \textbf{předem dohodnout}, že nechtějí jeden druhého zabít, a např. bojovat "na první krev"
\end{itemize}
Mezi znaky zkušeného šermíře patří schopnost být si vědom těchto mantinelů a v jejich rámci jednat kreativním a účinným způsobem.

Extrémní případ těchto omezení vidíme v moderních šermířských komunitách. U těch najdeme velmi \textbf{striktní pravidla} a vysoké nároky na bojovou výstroj s cílem zamezit jakýmkoliv vážným zraněním a ztrátám na životech, jelikož ty přirozeně nejsou cílem jejich snažení. 

Avšak pokud nedochází k žádným reálným zraněním, \textbf{jak lze rozhodnout, kdo vyhrál?} Zjevnou cestou je například systém zásahových zón a počítání životů, které se bojovníkovi v závislosti na zasažené zóně odečítají. Takový systém však nezaručuje realistický dojem z boje. Mezi plody snah o realističtější zážitek patří např. v české LARPové scéně oblíbený šatrh\cite{Satrh}, založený na důvěře v účastníky, že podle způsobu, jakým byli zasaženi, odhadnou a zahrají realistickou reakci, či buhurt, kde zkrátka vítězí ten, kdo nepadne vyčerpáním. Je tedy vidět, že i komunity, jejichž nejvyšší prioritou při boji je záruka bezpečí, jsou ochotné vyvinout značnou snahu, aby byl zážitek z boje realistický.

\subsection{Boj jeden na jednoho}
Jak tedy probíhá duel mezi dvěma šermíři? Hloubkový rozbor zohledňující historické i moderní techniky, rozdílné úrovně zkušenosti, zdatnosti a motivace obou zápasníků, veškeré obvyklé zbraně, kterými by mohli být vybaveni, prostředí a vnější okolnosti, které v souboji mohou hrát podstatnou roli, není možné provést v rozsahu bakalářské práce, natož jedné její kapitoly. Pro ten tedy čtenáře odkáži na specializovanou literaturu \cite{KunstDesFechtens} \cite{FightingWithTheGermanLongsword} \cite{ModernHEMA}.  

Zde si pouze nastíníme pár základních faktorů, které při souboji hrají důležitou roli.

\textbf{Obratnost zbraně} - schopnost rychle měnit její trajektorii, reagovat na změny situace (např. vyblokovat nečekaný úder) či změny situace sám vyvolat (např. na poslední chvíli zastavit fingovaný úder a stočit čepel na jiné místo, které protivník nechal nechráněné). Mezi faktory, kterými je determinována, patří velikost a hmotnost zbraně, její tvar, těžiště a způsob úchopu\footnote{Čím jsou ruce dál od sebe, tím silnější je princip páky.}; v neposlední řadě také síla a tělesné rozpoložení šermíře.

\textbf{Dosah, pohyb, terén} - V okamžiku, kdy vám protivník není schopen ublížit a vy jemu ano, můžete si dovolit být libovolně agresivní\footnote{Díky tomu dlouhé zbraně jako kopí mohou být velmi účinné i v rukou nezkušeného odvedence.}. Je-li váš protivník v takovéto výhodě, jedinou možností je se k němu přiblížit - pak se dostanete do výhody vy, jelikož zbraně s dlouhým dosahem zpravidla trpí na nízkou obratnost při boji zblízka\footnote{Nedejte protivníkovi čas tasit poboční zbraň.}. Vaše možnosti, jak se k němu přiblížit, přirozeně velmi závisí na terénu\footnote{Když stojíte pod hradbami a protivník bodá kopím z ochozu nad vámi, jste v pěkné kaši.}. 

Máte-li oba stejný dosah, pohyb se však nestává o nic méně důležitým. Správná technika pohybu pomáhá šermíři držet pod kontrolou jeho těžiště a nenechat se rozhodit, když klopýtne, či na něj protivník agresivně naběhne; případně mu umožní bleskově se přiblížit a využít nechráněnou oblast, kterou mu protivník nabídl svým nemotorným pohybem. 

\textbf{Zbroj}, kterou má protivník na sobě, může citelně omezit vaše možnosti co do způsobů, jak mu uštědřit zranění. Je-li například v plné plátové zbroji pozdního středověku, pokoušet se do ní sekat mečem je prakticky zbytečné. Vaše možnosti se citelně zúží - buď se můžete pokusit bodnout do některé ze skulin ve zbroji, nebo meč vyměnit za palcát či válečné kladivo\footnote{Případně lze meč chytit za čepel a záštitou zasuplovat hlavici kladiva - viz \cite{FightingWithTheGermanLongsword}} a protivníkovi tupými údery zlámat kosti. Protivník si vaší situace bude vědom, a o to jednodušší pro něj bude předvídat vaše akce.

Znakem každého dobrého šermíře je schopnost vcítit se do pozice svého protivníka a \textbf{odhadovat jeho budoucí akce} dříve, než se stanou. Nezkušený šermíř bude každý svůj úder provázet výrazným nápřahem, pro jeho protivníka pak není obtížné úder vyblokovat a přitom se rovnou k protivníkovi přiblížit a seknout po místě, které ví, že při úderu protivník nechal nechráněné. Zkušený šermíř pak ani nepotřebuje vidět výrazný nápřah, naučí se protivníkovu akci podvědomě vycítit z drobných náznaků (pohybů svalů, očí,...). Samozřejmě je možné předvádět údery fingovaně a čekat, že se při protiútoku otevře protivník vám. Pak záleží na jeho úsudku, zda úmysl prokoukne a provede fingovaný protiútok - stupňům falše se meze nekladou. 

\subsection{Hromadný boj a boj proti přesile}
Souboj jeden na jednoho však může vypadat jako téměř laboratorní podmínky vedle chaotických situací, v rámci kterých v historii běžně k boji docházelo.   

Často se střetávaly větší, více či méně \textbf{organizované skupiny}, případně celé armády čítající mnoho takových pevně ustanovených jednotek. V takové situaci je na místě, aby bojovníkova individualita ustoupila do pozadí - rozhodujícím faktorem je až na výjimky taktické jednání a sladěnost celku. 

Jedním ze základních požadavků, které jsou na vojáka kladeny, je schopnost \textbf{držení formace}. Stojíte-li jako semknutá řada, štít na štít, meče připravené opětovat jakýkoliv úder, který by protivník zkusil na kamaráda vedle, je mnohem těžší prorazit vašimi řadami a dostat se k lučištníkům, kteří nepříteli způsobují těžké ztráty. 

Při skupinové šarvátce hraje zpravidla rozhodující roli taktická \textbf{koordinace} spolubojovníků. Nepřítel, který vám překvapivě vpadl do zad, je mnohem nepříjemnější záležitost, než nepřítel, který postává opodál a čeká, až na něj přijde řada.  

Pro dosažení koordinace je však nutná \textbf{komunikace}. Dohodnout se při šarvátce na postupu, v omezeném čase a aniž by plán byl zachycen protivníkem, vskutku není triviální problém. V kontextu větší bitvy pak důležitost i obtížnost komunikace nabývá zcela nového rozměru a jde o těžký problém, který až do vynálezu moderní vysílačky nebyl uspokojivě vyřešen. Vojenská taktika si vyžaduje flexibilní reakce na změny situace na bojišti, informace musí proudit mezi vojenskými jednotkami a velitelstvím, které jsou často netriviálně fyzicky vzdálené. Jednou možností, jak informaci předat, je vyslání posla - ten je schopen nést detailní informaci, je však zranitelný a pomalý - než dorazí, může již jeho informace být neaktuální. Protipólem pak je broadcast předem dohodnutých signálů, napříkald prostřednictvím trubače - tato cesta je rychlá a v určitém doslechovém okruhu vcelku spolehlivá, avšak signály musí být předem dohodnuté a variant nemůže být mnoho, metoda tedy postrádá flexibilitu a detailnost informace. Jak vidno, jde o zásadní součást vedení bitvy, přesto však vidíme mnoho videoher, které ji zcela zanedbávají.
\bigbreak
Dojde-li k boji \textbf{jediného bojovníka proti přesile}, vše výše uvedené stále platí, akorát očividně kolosální výhody plynoucí z koordinovaného postupu jsou dopřány pouze jedné ze stran. I pro velmi zkušeného šermíře bojujícího proti několika nezkušeným, často jediná šance, jak z takové situace vyváznout, je protivníky rozdělit (kreativním použitím pohybu a znalosti terénu) a zdolat jednoho po druhém rychle, než mu ostatní přijdou na pomoc.

\subsection*{Shrnutí}
Seznámili jsme čtenáře s historickým pozadím a základními prvky, které charakterizují boj s chladnou zbraní v reálném světě. Nyní by měl být připraven pokročit dál a provést srovnání s přístupem, jakým tuto problematiku adaptuje svět videoher.

\section{Chladná zbraň v akčních videohrách}

Nyní, když máme rámcovou představu, jak boj s chladnou zbraní vypadá v reálném světě, nastíníme si, jakým způsobem bývá typicky adaptován v akčních videohrách - jaké jeho aspekty byly tímto universem věrně přejaty, jaké ignorovány, a jaké interpretovány s řádnou dávkou umělecké kreativity.

\subsection{Obecný přehled}

Účinkujícími ve hře jsou herní postavy. Ty mohou být ovládané buď hráčem, nebo počítačovým algoritmem, a lze je popsat nějakým jejich vnitřním stavem. Nyní si představíme základní veličiny, které jsou typickou součástí vnitřního stavu postavy bez ohledu, zda jde o hru zaměřenou na boj se střelnou zbraní, chladnou zbraní, či boj jakýkoliv jiný:

\begin{itemize}
    \item \textbf{\ac{HP}}... Číslo udávající zdravotní stav a životaschopnost postavy. Pokud klesne na nulu, postava umírá. Hráči jsou jeho HP obvykle v reálném čase explicitně komunikovány skrze ukazatel na obrazovce. V průběhu boje typicky neregeneruje\footnote{Popřípadě prostředky, skrze které ho lze doplnit, jsou vzácné.} a postava musí vystačit s tím, co má - jde tedy o zdroj vybízející k dlouhodobějšímu plánování.
    \item \textbf{Výdrž}... Volitelný avšak častý prvek, číslo reprezentující akutní schopnost postavy vynakládat fyzické úsilí. Spotřebováváno chvatným pohybem a prováděním akcí, v průběhu boje zpravidla dochází k jeho rychlému opakovanému spotřebování a regeneraci.
    \item \textbf{Inventář}... List dalších předmětů, které postava nese s sebou. Může jít o peníze, léčivé lektvary, munici do zbraní, náhradní zbraně, magické svitky, mapy a cokoliv dalšího v závislosti na hře. 
\end{itemize}

Postava v průběhu hry vykonává akce - akcí může být například pohyb, použití aktivní zbraně, interakce s okolím, či s předmětem z inventáře. Zde se podrobně zaměříme na akci použití zbraně.
  


\subsection{Zaklínač 3 - vzorový příklad}

\subsection{Chladná zbraň jako doplňkový prvek}



\section{Hry s větší kontrolou}

\subsection{Die by the Sword}

\subsection{Mount\&Blade}

\subsection{Kingdom Come: Deliverance}




\section{Alternativní ovladače}

\subsection{Virtuální realita}

\subsection{Pohybové senzory ve smartphonu}

\subsection{Obskurní periferie}


\section{Shrnutí}
