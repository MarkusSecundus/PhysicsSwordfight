\chapter{Úvod do problematiky}
V této kapitole uvedeme čtenáře do problematiky boje s chladnou zbraní - velmi stručně nastíníme jeho vývoj a metodiku v reálném světě, zkontrastujeme s jeho popkulturní reprezentací, a následně do hloubky rozebereme mechaniky, skrze které ho adaptuje svět videoher, a problémy, které při tom musí řešit.


\section{Chladná zbraň v reálném světě}
\subsection{Průlet historií}

Není nadsázkou když prohlásíme, že chladná zbraň je koncept starý jako lidstvo samo. Archelogické nálezy nasvědčují, že již vzdálený předek moderního člověka Australopithecus používal úderný předmět (pravděpodobně kus dřeva či kost ulovené antilopy) jako nástroj k lovu paviánů \cite{AustralopithecusWeapon}. Výhody, které mu to mohlo přinést, jsou zjevné: zatímco kořist měla možnosti obrany omezené tím, kam dosáhly její vlastní zuby a drápy, lovec si ji mohl držet v bezpečné vzdálenosti a vystavovat vlastní tělo o poznání menšímu nebezpečí odvety.

V průběhu svého vývoje lidský rod zdokonaloval i své zbraně, což přineslo mnoho dalších implikací - ostrý kamenný hrot člověku umožnil zasazovat ránu spolehlivě, a hlubší, než by kdy umožnilo pouhé jeho tělo. Spolehlivost zbraně umožnila lepší koordinaci mezi lovci. Sofistikovaná koordinace člověku otevřela cestu k lovu větších a silnějších druhů zvěře. Nakonec se člověk propracoval na samý vrchol potravního řetězce a zbraň, nástroj lovu, sloužila stále více pro vzájemný boj mezi lidmi. 

Zrod civilizací vznesl na zbraň nové požadavky. Vysoká koncentrace lidí na jednom místě vedla přirozeně ke specializaci profesí, mezi jinými i vojenské. Profesionální armáda čítající tisíce lidí umožňovala (a vyžadovala) do té doby nevídanou míru koordinace - ideálem v takové situaci se ukázalo disponovat širokou škálou zbraní vysoce specializovaných, chladných i střelných, jež mohly být použity ve vzájemné synergii, doplněné válečnými stroji a vhodně vycvičenými zvířaty\footnote{kůň jak známo hrál přední roli}. Profesionální voják měl čas a motivaci svůj typ zbraně pochopit do hloubky, stejně tak vysoce náročná si mohla dovolit být i její výroba a údržba, za níž nesli zodpovědnost rovněž profesionálové ve svém oboru. 
Na druhé straně tu však byl běžný obyvatel, který nepatřil k pravidelnému vojenskému jádru, ale byla-li říše pod útokem, považovalo se za samozřejmost, že pozvedne zbraň na její obranu. Zbraň pro takového člověka vyžadovala především jednoduchost - jak na výrobu, tak na údržbu a použití, zkrátka aby bylo možné v časové tísni krizové situace dostat co nejrychleji co nejvíce odvedenců do bojeschopného stavu a udržet je v něm.

Mocenské soupeření států ústí ve zběsilý a nikdy nekončící závod ve vývoji účinnějších zbraní a metod jejich použití. V jednom okamžiku dominuje Blízkému východu vozataj, v následujícím jej poráží Makedonská falanga, nad ní předvede svou superioritu Římský systém manipulů, ten vzápětí Římané sami prohlašují za zastaralý, avšak o půl tisíciletí později stejně jejich říši rozvrací hordy Hunských jezdců... chladná zbraň, ať už v rukou jezdce či pěšího vojáka, zůstává po většinu dějin dominantním prvkem na bojišti, se střelnými zbraněmi hrajícími významnou podpůrnou roli. Až s rozmachem zbraní palných se tato dynamika obrací a velmi pozvolně se dobíráme k modernímu vojenství. V současné době se chladná zbraň považuje převážně za překonanou - uplatnění pro ní stále existuje např. v kontextu pořádkových složek, avšak i pro ty plní úlohu spíše doplňkovou. Mimo oficiální kruhy pořád hraje nezanedbatelnou roli, avšak to je především pro její triviální dostupnost v porovnání s palnou zbraní.

Tradice chladné zbraně však stále žije v civilních komunitách dedikovaných zachování historie a kulturního odkazu. Mezi významné patří japonské umění Kendó ("cesta meče") vycházející ze samurajské tradice, či evropská komunita historického šermu, jež vychází z dochovaného učení středověkých mistrů. Rovněž je zde moderní sportovní šerm, jež přímo navazuje na šermířskou tradici ranného novověku. V posledních letech tyto vlivy více pronikají i do běžných volnočasových aktivit - ve Střední Evropě například je stále oblíbenější fenomén LARP\footnote{\ac{LARP}}, jehož jedna z podob - tzv. dřevárna - znamená hromadnou akci, na které se desítky až stovky lidí v tématických kostýmech a vyzbrojení zpravidla dřevěnými, molitanem měkčenými replikami zbraní, střetnou v bitvě. Pravidla boje přirozeně musí zaručit bezpečí účastníků, avšak stále při zachování autentického zážitku z boje.  

Vzhledem k charakteru a rozšířenosti výše zmíněných fenoménů lze očekávat, že jejich zastánci mají v nemalé míře zastoupení i ve videoherní komunitě. Přirozeně takoví lidé mohou mít zájem o hru, jež jim umožní jejich oblíbenou činnost napodobit, ale bez mnohých nepohodlí, jež ji provázejí v reálném světě. Pro hráče, který s bojem v reálném světě žádnou zkušenost nemá, může zase videohra být cestou, jak tuto mezeru ve svých zkušenostech částečně zaplnit a získat například hlubší porozumění k historii.

\subsection{Souboj dvou šermířů}



\subsection{Hromadný boj a boj proti přesile}


\section{Chladná zbraň ve videohrách}
Ve většině videoher je podoba boje oproti realitě silně stylizovaná, jak po vizuální stránce, tak z hlediska kontroly, která je hráči poskytnuta. 
Ovládání se zpravidla dá popsat tak, že hráč stiskne tlačítko, a postava sama zaútočí způsobem, jakým uzná za vhodné. Výhody, které tento přístup přináší:
\begin{itemize}
    \item Uchopitelnost pro nového hráče
    \item Uniformní rozhraní pro všechny typy zbraní
    \item 
\end{itemize}

\subsection{Zaklínač 3 - vzorový příklad}
q

\subsection{Chladná zbraň jako doplňkový prvek}
dsa



\section{Hry s větší kontrolou}
hh

\subsection{Die by the Sword}
gfd

\subsection{Mount\&Blade}
trq

\subsection{Kingdom Come: Deliverance}
bgz


\section{Alternativní ovladače}
uruzq

\subsection{Virtuální realita}
ghff

\subsection{Obskurní periferie}
hdj


\section{Shrnutí}
fd
