\chapter{Závěr}

Implementaci máme za sebou, prozkoumali jsme málo probádanou oblast akčních her s fyzikálně simulovaným, jemně ovladatelným mečem. V této sekci výsledek naší práce srovnáme s dalšími podobnými hrami, celkově ho zhodnotíme a nastínime možnosti jeho dalšího vývoje. 

\section{Srovnání s dalšími hrami}

V \ref{swordfightingIntroChapter}. kapitole jsme analyzovali několik her, které se podobně jako my pokusily poskytnout hráči přímou kontrolu (nejen) nad mečem. S těmi nyní náš výsledek srovnáme.

\subsection{Die by the Sword}

Tato hra je našemu systému nejbližší, v mnoha ohledech by se dala považovat za jeho nadmnožinu. Meč je zde rovněž fyzikálně simulovaným objektem, který se interně může pohybovat libovolně ve všech 6 stupních volnosti. 

\acl{DbtS} má mnohem propracovanější systémy umělé inteligence nepřátel a procedurální animace, díky kterým je umožněno např. \textbf{odsekávání končetin nepřátel}, kteří na svůj nový stav adekvátně reagují. Jádro našeho systému budoucí implementaci takové mechaniky nijak citelně nebrání, avšak i tak by vyžadovala mnoho práce a některé části herní logiky by musely být změněny. 

\acl{DbtS} zahrnuje větší \textbf{množství nepřátel} - každého se značně odlišnou tělesnou stavbou a používajícího zbraň jiným způsobem. Naše ukázková hra mnoho nepřátel nezahrnuje, v budoucnu by však nemělo být obtížné je přidat. Jádro našeho systému tomu nijak nebrání - stačí pouze implementovat daného nepřítele, hráče ani ostatní \acs{NPC}s není třeba upravovat, aby s novým nepřítelem byli schopni zápasit.  

Použití meče v obou hrách nabízí srovnatelně velké možnosti \textbf{emergentního gameplaye}. Naše hra představením konceptu přepínatelných \textit{modů ovládání} poskytuje větší prostor pro tvorbu sofistikovaných technik použití meče, \acl{DbtS} to však vynahrazuje kombinací s dalšími mechanikami, jakými je např. výše zmíněné odsekávání nepřátelských končetin. 

Volnost ovládání meče rovněž v obou hrách poskytuje srovnatelné, velmi značné možnosti \textbf{interakce s okolním prostředím}. Level design \acl{DbtS} však tohoto valně nevyužívá - ani základní aktivace tlačítek není vykonávána pomocí meče, ale jako specielní akce.

\textbf{Ovládání pohybu postavy} je v obou hrách řešeno prakticky stejně (WASD pro pohyb dopředu a do stran, horizontální rotace pomocí dalších tlačítek). Není zcela ideální a vyžaduje čas, aby si na něj hráč zvykl.

\textbf{Ovládání meče} je oblast, kde lze naší hru považovat za nadmnožinu \acl{DbtS} - kromě sekání podporujeme navíc mod blokování a jednoduše lze doimplementovat další. \acl{DbtS} sice narozdíl od nás podporuje přehrávání nahrávek, avšak ty v praxi nejsou zcela použitelné, do naší hry by navíc nebylo obtížné je rovněž přidat. \texttt{IRayIntersectable}s v naší hře nabízejí další vrstvu logiky ovládání meče, která je bohatě konfigurovatelná. Aby jí však typický hráč mohl využívat, bylo by třeba implementovat uživatelsky přívětivý in-game editor.  

Po \textbf{technické stránce} má \acl{DbtS} výhodu, že používá vlastní fyzikální engine plně šitý na míru jeho požadavkům. Detekce kolizí mečů tedy funguje korektně ve výchozím stavu a její overhead neroste kvadraticky s počtem mečů ve scéně jako u nás. 



\subsection{Mount\&Blade}

V této hře najdeme větší množství zbraní než pouze meč (palcáty, kopí, hole,...), ty se však neúčastní plnohodnotně fyzikální simulace a možnosti jejich ovládání jsou omezené na vcelku malou, konečnou množinu akcí. 

Najdeme zde \textbf{systém dílů zbroje}, které poskytují různou ochranu různým částem těla. Náš systém obsahuje základní stavební bloky pro implementaci podobné mechaniky.

\textbf{Ovládání rotace kamery} činí problém pro obě hry. V našem systému má však na hratelnost citelnější dopad (výběr útokové zóny nevyžaduje takovou preciznost jako přímé udávání směru meče).

Pohyb mečem je v Mount\&Blade umožněn pouze v několika předem určených směrech - nad mečem hráči není poskytnuta dostatečně volná kontrola, aby vznikl potenciál pro \textbf{emergentní gameplay}. Zdroje emergentního gameplaye však najdeme v dalších mechanikách - především v koordinaci podřízených vojáků. 

Pro \textbf{interakci s objekty okolního prostředí} zbraň poskytuje mizivý potenciál. Level design hry se ho nepokouší využít.

Narozdíl od nás, Mount\&Blade je zaměřené čistě na boj s \textbf{humanoidními protivníky} - přidání entity s jakkoliv exotičtější stavbou těla by vyžadovalo překopání všech ostatních entit, aby proti té nové byly schopny účinně bojovat.

\bigbreak

Jednou z ústředních mechanik Mount\&Blade je \textbf{velení podřízeným bojovníkům}. Jde o mechaniku ortogonální k ovládání meče - v našem systému by ji šlo implementovat rovněž, jeho současná implementace by takovým snahám nepostavila do cesty žádné překážky, ani by je nijak nepodpořila.

Pro umožnění hromadných bojů by však bylo třeba provést další optimalizace našeho systému korekce kolizí, které by zredukovaly jeho overhead na subkvadratický vůči počtu mečů ve scéně. 




\subsection{Kingdom Come: Deliverance}



\begin{itemize}
    \item taky konečný počet akcí co lze dělat
    \item systém výdrže - bylo by velmi zajímavé zakomponovat u nás, ale bude to těžké (jak chceme determinovat výdrž k ubrání za vykonání útoku?)
    \item 
    \item časovací eventy - náš systém neumožňuje
    \item RPG progrese - velká část by se dala implementovat i u nás (úpravou parametrů aby meč sekal míň/víc elegantně, odemykáním nových modů ovládání), ale komba úplně nepůjdou
    \item ovládání kamery - mohlo by být zajímavé zkusit lock-on u nás
    \item interakce s prostředím - vůbec
    \item emergentní gameplay (?) 
\end{itemize}

\section{Zhodnocení}

\section{Možnosti rozšíření}

\begin{itemize}
    \item \textbf{Audio}
    \item \textbf{Grafika}
    \item \textbf{Další mody ovládání}
    \item \textbf{Podpora ovladače Wii Remote}
    \item \textbf{Přepracování IRayIntersectable}
    \item \textbf{Méně výpočetně náročná korekce kolizí}
    \item \textbf{Sofistikovanější nepřátelská AI}
    \item \textbf{Kombinace s dalšími herními mechanikami} - výdrž, RPG progrese
    \item \textbf{Rozšíření na hru plné velikosti}
\end{itemize}