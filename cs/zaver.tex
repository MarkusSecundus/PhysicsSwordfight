\chapter{Závěr}

Implementaci máme za sebou, prozkoumali jsme málo probádanou oblast akčních her s fyzikálně simulovaným, jemně ovladatelným mečem. V této sekci výsledek naší práce srovnáme s dalšími podobnými hrami, celkově ho zhodnotíme a nastínime možnosti jeho dalšího vývoje. 

\section{Srovnání s dalšími hrami}

V \ref{swordfightingIntroChapter}. kapitole jsme analyzovali několik her, které se, podobně jako my, pokusily poskytnout hráči přímou kontrolu (nejen) nad mečem. S těmi nyní náš výsledek srovnáme.

\subsection{Die by the Sword}

Tato hra je našemu systému nejbližší, v mnoha ohledech by se dala považovat za jeho nadmnožinu. Meč je zde rovněž fyzikálně simulovaným objektem, který se interně může pohybovat libovolně ve všech 6 stupních volnosti. 

\acl{DbtS} má mnohem propracovanější systémy umělé inteligence nepřátel a procedurální animace, díky kterým je umožněno např. \textbf{odsekávání končetin nepřátel}, kteří na svůj nový stav adekvátně reagují. Jádro našeho systému budoucí implementaci takové mechaniky nijak citelně nebrání, avšak i tak by vyžadovala mnoho práce a některé části herní logiky by musely být změněny. 

\acl{DbtS} zahrnuje větší \textbf{množství nepřátel} - každého se značně odlišnou tělesnou stavbou a používajícího zbraň jiným způsobem. Naše ukázková hra mnoho nepřátel nezahrnuje, v budoucnu by však nemělo být obtížné je přidat. Jádro našeho systému tomu nijak nebrání - stačí pouze implementovat daného nepřítele, hráče ani ostatní \acs{NPC}s není třeba upravovat, aby s novým nepřítelem byli schopni zápasit.  

Použití meče v obou hrách nabízí srovnatelně velké možnosti \textbf{emergentního gameplaye}. Naše hra představením konceptu přepínatelných \textit{modů ovládání} poskytuje větší prostor pro tvorbu sofistikovaných technik použití meče, \acl{DbtS} to však vynahrazuje kombinací s dalšími mechanikami, jakými je např. výše zmíněné odsekávání nepřátelských končetin. 

Volnost ovládání meče rovněž v obou hrách poskytuje srovnatelné, velmi značné možnosti \textbf{interakce s okolním prostředím}. Level design \acl{DbtS} však tohoto valně nevyužívá - ani základní aktivace tlačítek není vykonávána pomocí meče, ale jako specielní akce.

\textbf{Ovládání pohybu postavy} je v obou hrách řešeno prakticky stejně. Není zcela ideální a vyžaduje čas, aby si na něj hráč zvykl.

\textbf{Ovládání meče} je oblast, kde lze naší hru považovat za nadmnožinu \acl{DbtS} - kromě sekání podporujeme navíc mod blokování a jednoduše lze doimplementovat další. \acl{DbtS} sice narozdíl od nás podporuje přehrávání nahrávek, avšak to v praxi není valně použitelná metoda ovládání, do naší hry by navíc nebylo obtížné je rovněž přidat. \texttt{IRayIntersectable}s v naší hře nabízejí další vrstvu logiky ovládání meče, která je bohatě konfigurovatelná. Aby jí však typický hráč mohl využívat, bylo by třeba implementovat uživatelsky přívětivý in-game editor.  

Po \textbf{technické stránce} má \acl{DbtS} výhodu, že používá vlastní fyzikální engine plně šitý na míru jeho požadavkům. Detekce kolizí mečů tedy funguje korektně ve výchozím stavu a její overhead neroste kvadraticky s počtem mečů ve scéně jako u nás. 



\subsection{Mount\&Blade}

V této hře najdeme větší množství zbraní než pouze meč (palcáty, kopí, hole,...), ty se však neúčastní plnohodnotně fyzikální simulace a možnosti jejich ovládání jsou omezené na vcelku malou, konečnou množinu akcí. Některé zbraně umožňují bodat, jinak jsou však proveditelné akce podmnožinou toho, co umožňuje náš systém.

Najdeme zde \textbf{systém dílů zbroje}, které poskytují různou ochranu různým částem těla. Náš systém obsahuje základní stavební bloky pro implementaci podobné mechaniky.

\textbf{Ovládání rotace kamery} činí problém pro obě hry. V našem systému má však na hratelnost citelnější dopad (výběr útokové zóny nevyžaduje takovou preciznost ovládání jako přímé udávání směru meče).

Pohyb mečem je v Mount\&Blade umožněn pouze v několika předem určených směrech - nad mečem hráči není poskytnuta dostatečně volná kontrola, aby vznikl potenciál pro \textbf{emergentní gameplay}. Zdroje emergentního gameplaye však najdeme v dalších mechanikách - především v koordinaci podřízených vojáků. 

Pro \textbf{interakci s objekty okolního prostředí} zbraň poskytuje mizivý potenciál. Hra ani žádné objekty okolního prostředí, se kterými by bylo možné jakkoliv netriviálně interagovat, neobsahuje.

Narozdíl od nás, Mount\&Blade je zaměřené čistě na boj s \textbf{humanoidními protivníky} - přidání entity s jakkoliv exotičtější stavbou těla by vyžadovalo překopání všech ostatních entit, aby proti té nové byly schopny účinně bojovat.


Jednou z ústředních mechanik Mount\&Blade je \textbf{velení podřízeným bojovníkům}. Jde o mechaniku ortogonální k ovládání meče - v našem systému by ji šlo implementovat rovněž, jeho současná implementace by takovým snahám nepostavila do cesty žádné překážky, ani by je nijak nepodpořila.

Pro umožnění hromadných bojů by však bylo třeba provést další optimalizace naší metody korekce kolizí, které by zredukovaly její overhead na subkvadratický vůči počtu mečů ve scéně. 




\subsection{Kingdom Come: Deliverance} \label{finalComparisonWithKCD}

Zde platí mnoho stejných pozorování jako u Mount\&Blade - rovněž tu najdeme větší množství zbraní, které se neúčastní celkové fyzikální simulace a lze s nimi vykonávat pouze konečné množství typů akcí. Vše je striktně šité na míru boji proti humanoidům a možnosti interakce s okolním prostředím jsou nulové.

Blokování je zde samostatná akce založená na \textbf{prvcích časování}. Náš systém naproti tomu klasické časovací mechaniky z principu věci nepodporuje.

\textbf{Ovládání kamery} je v \acl{KCD} vyřešené velmi dobře. Mohlo by být zajímavé se zde inspirovat a zkusit zakomponovat uzamčení kamery na protivníka i do našeho systému.

Velmi výrazným prvkem, který boji dodává značnou hloubku, je v \acl{KCD} \textbf{systém výdrže}. Přidání podobné mechaniky by mohlo značně obohatit i náš bojový systém, implementace by však nebyla zcela přímočará (např. by byl netrivální problém stanovit kolik výdrže se hráči má odečíst za vykonaný útok).

Dalším prvkem, kterým je bojový systém v \acl{KCD} význačný, je jeho spjatost se systémem \textbf{\acs{RPG} progrese} - postava, která drží meč poprvé v životě, s ním zachází nemotorně, kdežto mistr šermíř zasazuje elegantní údery. Tato stránka věci by v našem systému mohla být na implementaci překvapivě snadná - stačilo by pro každý stupeň dovednosti najít adekvátní fyzikální parametry (viz \ref{swordParameterTweaksSection}). V \acl{KCD} však jsou s progresí rovněž spjata komba\footnote{specifické sekvence úderů, které na konci vyvolají specielní akci} - ty náš systém z podstaty věci není schopen umožnit.  

Velmi silnou stránkou \acl{KCD} je rovněž \textbf{vizuální prezentace}. Díky tomu, že akcí, které lze vykonat, je pouze pevně dané konečné množství, bylo možné pro každou z nich do detailu ručně vypilovat řetězec animací, ze kterých jsou poskládané. V našem systému, který z velké části vyžaduje procedurální animaci, z principu věci nelze srovnatelné umělecké preciznosti dosáhnout. 


\section{Zhodnocení}

Naše práce ověřila, že za použití moderního fyzikálního systému, jakým je ten vestavěný v enginu Unity, je velmi dobře možné implementovat akční hru s mečem, která hráči poskytne nad zbraní jemnou kontrolu ve všech šesti stupních volnosti. Jádro takové hry jsme implementovali a může posloužit jako základ pro plnohodnotnou komerční hru či jako výchozí bod pro další výzkum v této oblasti.




\section{Možnosti rozšíření}

\begin{itemize}
    \item \textbf{Audio} - v současném stavu hra neobsahuje žádnou zvukovou složku. Je třeba zkomponovat hudební podkres a přidat zvuky vydávané herními entitami. Generování fyzikálně realistických zvuků meče by mohlo být zajímavou oblastí pro další výzkum. 
    \item \textbf{Grafika} - vizuální stránka hry vyžaduje další práci. Bylo by vhodné pro šermíře vytvořit animace pohybu, pro všechny nepřátele pak dokončit efekt krvavých skvrn reflektujících stav jejich \acs{HP}.
    \item \textbf{Další mody ovládání} - poskytnout výchozí bod pro navazující výzkum v oblasti ovládání meče byl jeden z primárních cílů této práce. Mody, které hra současně podporuje, již nabízejí vcelku flexibilní možnosti ovládání, stále však najdeme očividné směry, které vybízejí k dalšímu rozšiřování - např. bodání s mečem. 
    \item \textbf{Podpora ovladače Wii Remote} - ovladač, který by nám umožnil přímou kontrolu nad pohybem meče ve všech šesti stupních volnosti, by mohl posloužit jako cenný kontrolní bod pro navazující experimenty v oblasti ovládání meče tradičními periferiemi. Ovladač Wii Remote konkrétně je možností spolehlivou, finančně dostupnou a použitelnou v enginu Unity (viz \ref{wiiRemoteSubsection}).
    \item \textbf{Přepracování IRayIntersectable} - v procesu mapování uživatelského vstupu na pohyb meče je nejprve třeba z paprsku vystřeleného z polohy hráčova kurzoru získat kontrolní bod. Toho dosahujeme průnikem paprsku s geometrickým útvarem. Současný systém (viz \ref{rayIntersectablesDefinitionSubsection}) umožňuje definovat tyto útvary vcelku flexibilně, ale nese neduhy v podobě špatných možností vizualizace a značné míry ručního ladění potřebné k dosažení dobrého výsledku. Rovněž chybí editor, s jehož pomocí by tyto útvary mohl editovat hráč uvnitř hry.
    \item \textbf{Lepší ovládání pohybu postavy} - jak se zdá, precizní ovládání meče vyžaduje, aby byla myš vyhrazena plně pro něj. Ovládání pohybu postavy tedy musí vystačit s klávesnicí - přímočaré ovladací schéma, jaké naše práce použila, nevede ke zcela optimálnímu hráčskému zážitku. Stejně jako v oblasti ovládání meče je zde třeba další výzkum.
    \item \textbf{Méně výpočetně náročná korekce kolizí} - metoda (viz \ref{swordCollisionsSection}), kterou jsme použili, abychom zabránili tunelování a celkově nekorektnímu chování mečů při vzájemných kolizích, nese overhead, který roste v nejhorším případě kvadraticky s počtem mečů ve scéně. Pro umožnění hromadných bitev je třeba nalézt preciznější heuristiky, s jejichž pomocí by pomocné collidery byly poolovány a umisťovány pouze tam, kde jsou skutečně třeba.
    \item \textbf{Balancing} - je třeba provést testování na dobrovolnících a podchytit 
    \item \textbf{Multiplayer} - umožnit vzájemný boj několika hráčů se pro náší hru jeví jako přirozený další krok.  
    \item \textbf{Sofistikovanější nepřátelská AI} - algoritmus, který jsme použili pro ovládání nepřátelských šermířů, je velmi jednoduchý a pro hráče neposkytuje značnou výzvu. Vytvořit umělou inteligenci schopnou využít plně možností, jež jí jemná kontrola nad mečem poskytuje, by mohlo být zajímavým úkolem pro další výzkum - nabízí se např. zapojení evolučních algoritmů.
    \item \textbf{Výzkum ohledně interakce s prostředím}
    \item \textbf{Kombinace s dalšími herními mechanikami} - mnoho her přidává svému bojovému systému dodatečnou hloubku kombinací s dalšími mechanikami jako je systém výdrže, magie či \acs{RPG} progrese.
    \item \textbf{Rozšíření na hru plné velikosti} - jádro, které jsme vytvořili, je třeba doplnit dalšími ortogonálními mechanikami, přidat zajímavé nepřátele, narativ a level design, aby bylo dosaženo hry, která naplní standardy komerčního herního trhu. 
\end{itemize}