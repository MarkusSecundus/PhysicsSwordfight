%%% Šablona pro jednoduchý soubor formátu PDF/A, jako treba samostatný abstrakt práce.

\documentclass[12pt]{report}

\usepackage[a4paper, hmargin=1in, vmargin=1in]{geometry}
\usepackage[a-2u]{pdfx}
\usepackage[czech]{babel}
\usepackage[utf8]{inputenc}
\usepackage[T1]{fontenc}
\usepackage{lmodern}
\usepackage{textcomp}

\begin{document}

%% Nezapomeňte upravit abstrakt.xmpdata.

Boj nablízko s chladnou zbraní najdeme v mnoha videohrách, pouze hrstka z nich se však pokouší o realistickou simulaci, jež by hráči dala svobodu jakkoliv se blížící té, již manipulace s chladnou zbraní umožňuje v reálném světě. Očividným těžkým problémem při takových snahách je návrh schématu ovládání pro klasické počítačové periferie. Ještě méně prozkoumanou oblastí je pak zapojení zbraně do fyzikální simulace.

Práce implementuje simulátor šermu s jedenapůlručním mečem v enginu Unity, v němž figuruje meč jako plně fyzikálně simulovaný objekt. Rovněž předkládá schéma ovládání pro klávesnici a myš umožňující hráči jemnou kontrolu nad pohybem zbraně. Pro testování hratelnosti implementuje jednoduchého AI protivníka.
Implementace je vytvořena s použitím dobrých programátorských praktik a může komukoliv posloužit jako základ pro akční hru s pokročilým bojovým systémem.

\end{document}
